
% Default to the notebook output style

    


% Inherit from the specified cell style.




    
\documentclass[11pt]{article}

    
    
    \usepackage[T1]{fontenc}
    % Nicer default font than Computer Modern for most use cases
    \usepackage{palatino}

    % Basic figure setup, for now with no caption control since it's done
    % automatically by Pandoc (which extracts ![](path) syntax from Markdown).
    \usepackage{graphicx}
    % We will generate all images so they have a width \maxwidth. This means
    % that they will get their normal width if they fit onto the page, but
    % are scaled down if they would overflow the margins.
    \makeatletter
    \def\maxwidth{\ifdim\Gin@nat@width>\linewidth\linewidth
    \else\Gin@nat@width\fi}
    \makeatother
    \let\Oldincludegraphics\includegraphics
    % Set max figure width to be 80% of text width, for now hardcoded.
    \renewcommand{\includegraphics}[1]{\Oldincludegraphics[width=.8\maxwidth]{#1}}
    % Ensure that by default, figures have no caption (until we provide a
    % proper Figure object with a Caption API and a way to capture that
    % in the conversion process - todo).
    \usepackage{caption}
    \DeclareCaptionLabelFormat{nolabel}{}
    \captionsetup{labelformat=nolabel}

    \usepackage{adjustbox} % Used to constrain images to a maximum size 
    \usepackage{xcolor} % Allow colors to be defined
    \usepackage{enumerate} % Needed for markdown enumerations to work
    \usepackage{geometry} % Used to adjust the document margins
    \usepackage{amsmath} % Equations
    \usepackage{amssymb} % Equations
    \usepackage{textcomp} % defines textquotesingle
    % Hack from http://tex.stackexchange.com/a/47451/13684:
    \AtBeginDocument{%
        \def\PYZsq{\textquotesingle}% Upright quotes in Pygmentized code
    }
    \usepackage{upquote} % Upright quotes for verbatim code
    \usepackage{eurosym} % defines \euro
    \usepackage[mathletters]{ucs} % Extended unicode (utf-8) support
    \usepackage[utf8x]{inputenc} % Allow utf-8 characters in the tex document
    \usepackage{fancyvrb} % verbatim replacement that allows latex
    \usepackage{grffile} % extends the file name processing of package graphics 
                         % to support a larger range 
    % The hyperref package gives us a pdf with properly built
    % internal navigation ('pdf bookmarks' for the table of contents,
    % internal cross-reference links, web links for URLs, etc.)
    \usepackage{hyperref}
    \usepackage{longtable} % longtable support required by pandoc >1.10
    \usepackage{booktabs}  % table support for pandoc > 1.12.2
    \usepackage[normalem]{ulem} % ulem is needed to support strikethroughs (\sout)
                                % normalem makes italics be italics, not underlines
    

    
    
    % Colors for the hyperref package
    \definecolor{urlcolor}{rgb}{0,.145,.698}
    \definecolor{linkcolor}{rgb}{.71,0.21,0.01}
    \definecolor{citecolor}{rgb}{.12,.54,.11}

    % ANSI colors
    \definecolor{ansi-black}{HTML}{3E424D}
    \definecolor{ansi-black-intense}{HTML}{282C36}
    \definecolor{ansi-red}{HTML}{E75C58}
    \definecolor{ansi-red-intense}{HTML}{B22B31}
    \definecolor{ansi-green}{HTML}{00A250}
    \definecolor{ansi-green-intense}{HTML}{007427}
    \definecolor{ansi-yellow}{HTML}{DDB62B}
    \definecolor{ansi-yellow-intense}{HTML}{B27D12}
    \definecolor{ansi-blue}{HTML}{208FFB}
    \definecolor{ansi-blue-intense}{HTML}{0065CA}
    \definecolor{ansi-magenta}{HTML}{D160C4}
    \definecolor{ansi-magenta-intense}{HTML}{A03196}
    \definecolor{ansi-cyan}{HTML}{60C6C8}
    \definecolor{ansi-cyan-intense}{HTML}{258F8F}
    \definecolor{ansi-white}{HTML}{C5C1B4}
    \definecolor{ansi-white-intense}{HTML}{A1A6B2}

    % commands and environments needed by pandoc snippets
    % extracted from the output of `pandoc -s`
    \providecommand{\tightlist}{%
      \setlength{\itemsep}{0pt}\setlength{\parskip}{0pt}}
    \DefineVerbatimEnvironment{Highlighting}{Verbatim}{commandchars=\\\{\}}
    % Add ',fontsize=\small' for more characters per line
    \newenvironment{Shaded}{}{}
    \newcommand{\KeywordTok}[1]{\textcolor[rgb]{0.00,0.44,0.13}{\textbf{{#1}}}}
    \newcommand{\DataTypeTok}[1]{\textcolor[rgb]{0.56,0.13,0.00}{{#1}}}
    \newcommand{\DecValTok}[1]{\textcolor[rgb]{0.25,0.63,0.44}{{#1}}}
    \newcommand{\BaseNTok}[1]{\textcolor[rgb]{0.25,0.63,0.44}{{#1}}}
    \newcommand{\FloatTok}[1]{\textcolor[rgb]{0.25,0.63,0.44}{{#1}}}
    \newcommand{\CharTok}[1]{\textcolor[rgb]{0.25,0.44,0.63}{{#1}}}
    \newcommand{\StringTok}[1]{\textcolor[rgb]{0.25,0.44,0.63}{{#1}}}
    \newcommand{\CommentTok}[1]{\textcolor[rgb]{0.38,0.63,0.69}{\textit{{#1}}}}
    \newcommand{\OtherTok}[1]{\textcolor[rgb]{0.00,0.44,0.13}{{#1}}}
    \newcommand{\AlertTok}[1]{\textcolor[rgb]{1.00,0.00,0.00}{\textbf{{#1}}}}
    \newcommand{\FunctionTok}[1]{\textcolor[rgb]{0.02,0.16,0.49}{{#1}}}
    \newcommand{\RegionMarkerTok}[1]{{#1}}
    \newcommand{\ErrorTok}[1]{\textcolor[rgb]{1.00,0.00,0.00}{\textbf{{#1}}}}
    \newcommand{\NormalTok}[1]{{#1}}
    
    % Additional commands for more recent versions of Pandoc
    \newcommand{\ConstantTok}[1]{\textcolor[rgb]{0.53,0.00,0.00}{{#1}}}
    \newcommand{\SpecialCharTok}[1]{\textcolor[rgb]{0.25,0.44,0.63}{{#1}}}
    \newcommand{\VerbatimStringTok}[1]{\textcolor[rgb]{0.25,0.44,0.63}{{#1}}}
    \newcommand{\SpecialStringTok}[1]{\textcolor[rgb]{0.73,0.40,0.53}{{#1}}}
    \newcommand{\ImportTok}[1]{{#1}}
    \newcommand{\DocumentationTok}[1]{\textcolor[rgb]{0.73,0.13,0.13}{\textit{{#1}}}}
    \newcommand{\AnnotationTok}[1]{\textcolor[rgb]{0.38,0.63,0.69}{\textbf{\textit{{#1}}}}}
    \newcommand{\CommentVarTok}[1]{\textcolor[rgb]{0.38,0.63,0.69}{\textbf{\textit{{#1}}}}}
    \newcommand{\VariableTok}[1]{\textcolor[rgb]{0.10,0.09,0.49}{{#1}}}
    \newcommand{\ControlFlowTok}[1]{\textcolor[rgb]{0.00,0.44,0.13}{\textbf{{#1}}}}
    \newcommand{\OperatorTok}[1]{\textcolor[rgb]{0.40,0.40,0.40}{{#1}}}
    \newcommand{\BuiltInTok}[1]{{#1}}
    \newcommand{\ExtensionTok}[1]{{#1}}
    \newcommand{\PreprocessorTok}[1]{\textcolor[rgb]{0.74,0.48,0.00}{{#1}}}
    \newcommand{\AttributeTok}[1]{\textcolor[rgb]{0.49,0.56,0.16}{{#1}}}
    \newcommand{\InformationTok}[1]{\textcolor[rgb]{0.38,0.63,0.69}{\textbf{\textit{{#1}}}}}
    \newcommand{\WarningTok}[1]{\textcolor[rgb]{0.38,0.63,0.69}{\textbf{\textit{{#1}}}}}
    
    
    % Define a nice break command that doesn't care if a line doesn't already
    % exist.
    \def\br{\hspace*{\fill} \\* }
    % Math Jax compatability definitions
    \def\gt{>}
    \def\lt{<}
    % Document parameters
    \title{Trigger\_word\_detection\_v1a}
    
    
    

    % Pygments definitions
    
\makeatletter
\def\PY@reset{\let\PY@it=\relax \let\PY@bf=\relax%
    \let\PY@ul=\relax \let\PY@tc=\relax%
    \let\PY@bc=\relax \let\PY@ff=\relax}
\def\PY@tok#1{\csname PY@tok@#1\endcsname}
\def\PY@toks#1+{\ifx\relax#1\empty\else%
    \PY@tok{#1}\expandafter\PY@toks\fi}
\def\PY@do#1{\PY@bc{\PY@tc{\PY@ul{%
    \PY@it{\PY@bf{\PY@ff{#1}}}}}}}
\def\PY#1#2{\PY@reset\PY@toks#1+\relax+\PY@do{#2}}

\expandafter\def\csname PY@tok@w\endcsname{\def\PY@tc##1{\textcolor[rgb]{0.73,0.73,0.73}{##1}}}
\expandafter\def\csname PY@tok@c\endcsname{\let\PY@it=\textit\def\PY@tc##1{\textcolor[rgb]{0.25,0.50,0.50}{##1}}}
\expandafter\def\csname PY@tok@cp\endcsname{\def\PY@tc##1{\textcolor[rgb]{0.74,0.48,0.00}{##1}}}
\expandafter\def\csname PY@tok@k\endcsname{\let\PY@bf=\textbf\def\PY@tc##1{\textcolor[rgb]{0.00,0.50,0.00}{##1}}}
\expandafter\def\csname PY@tok@kp\endcsname{\def\PY@tc##1{\textcolor[rgb]{0.00,0.50,0.00}{##1}}}
\expandafter\def\csname PY@tok@kt\endcsname{\def\PY@tc##1{\textcolor[rgb]{0.69,0.00,0.25}{##1}}}
\expandafter\def\csname PY@tok@o\endcsname{\def\PY@tc##1{\textcolor[rgb]{0.40,0.40,0.40}{##1}}}
\expandafter\def\csname PY@tok@ow\endcsname{\let\PY@bf=\textbf\def\PY@tc##1{\textcolor[rgb]{0.67,0.13,1.00}{##1}}}
\expandafter\def\csname PY@tok@nb\endcsname{\def\PY@tc##1{\textcolor[rgb]{0.00,0.50,0.00}{##1}}}
\expandafter\def\csname PY@tok@nf\endcsname{\def\PY@tc##1{\textcolor[rgb]{0.00,0.00,1.00}{##1}}}
\expandafter\def\csname PY@tok@nc\endcsname{\let\PY@bf=\textbf\def\PY@tc##1{\textcolor[rgb]{0.00,0.00,1.00}{##1}}}
\expandafter\def\csname PY@tok@nn\endcsname{\let\PY@bf=\textbf\def\PY@tc##1{\textcolor[rgb]{0.00,0.00,1.00}{##1}}}
\expandafter\def\csname PY@tok@ne\endcsname{\let\PY@bf=\textbf\def\PY@tc##1{\textcolor[rgb]{0.82,0.25,0.23}{##1}}}
\expandafter\def\csname PY@tok@nv\endcsname{\def\PY@tc##1{\textcolor[rgb]{0.10,0.09,0.49}{##1}}}
\expandafter\def\csname PY@tok@no\endcsname{\def\PY@tc##1{\textcolor[rgb]{0.53,0.00,0.00}{##1}}}
\expandafter\def\csname PY@tok@nl\endcsname{\def\PY@tc##1{\textcolor[rgb]{0.63,0.63,0.00}{##1}}}
\expandafter\def\csname PY@tok@ni\endcsname{\let\PY@bf=\textbf\def\PY@tc##1{\textcolor[rgb]{0.60,0.60,0.60}{##1}}}
\expandafter\def\csname PY@tok@na\endcsname{\def\PY@tc##1{\textcolor[rgb]{0.49,0.56,0.16}{##1}}}
\expandafter\def\csname PY@tok@nt\endcsname{\let\PY@bf=\textbf\def\PY@tc##1{\textcolor[rgb]{0.00,0.50,0.00}{##1}}}
\expandafter\def\csname PY@tok@nd\endcsname{\def\PY@tc##1{\textcolor[rgb]{0.67,0.13,1.00}{##1}}}
\expandafter\def\csname PY@tok@s\endcsname{\def\PY@tc##1{\textcolor[rgb]{0.73,0.13,0.13}{##1}}}
\expandafter\def\csname PY@tok@sd\endcsname{\let\PY@it=\textit\def\PY@tc##1{\textcolor[rgb]{0.73,0.13,0.13}{##1}}}
\expandafter\def\csname PY@tok@si\endcsname{\let\PY@bf=\textbf\def\PY@tc##1{\textcolor[rgb]{0.73,0.40,0.53}{##1}}}
\expandafter\def\csname PY@tok@se\endcsname{\let\PY@bf=\textbf\def\PY@tc##1{\textcolor[rgb]{0.73,0.40,0.13}{##1}}}
\expandafter\def\csname PY@tok@sr\endcsname{\def\PY@tc##1{\textcolor[rgb]{0.73,0.40,0.53}{##1}}}
\expandafter\def\csname PY@tok@ss\endcsname{\def\PY@tc##1{\textcolor[rgb]{0.10,0.09,0.49}{##1}}}
\expandafter\def\csname PY@tok@sx\endcsname{\def\PY@tc##1{\textcolor[rgb]{0.00,0.50,0.00}{##1}}}
\expandafter\def\csname PY@tok@m\endcsname{\def\PY@tc##1{\textcolor[rgb]{0.40,0.40,0.40}{##1}}}
\expandafter\def\csname PY@tok@gh\endcsname{\let\PY@bf=\textbf\def\PY@tc##1{\textcolor[rgb]{0.00,0.00,0.50}{##1}}}
\expandafter\def\csname PY@tok@gu\endcsname{\let\PY@bf=\textbf\def\PY@tc##1{\textcolor[rgb]{0.50,0.00,0.50}{##1}}}
\expandafter\def\csname PY@tok@gd\endcsname{\def\PY@tc##1{\textcolor[rgb]{0.63,0.00,0.00}{##1}}}
\expandafter\def\csname PY@tok@gi\endcsname{\def\PY@tc##1{\textcolor[rgb]{0.00,0.63,0.00}{##1}}}
\expandafter\def\csname PY@tok@gr\endcsname{\def\PY@tc##1{\textcolor[rgb]{1.00,0.00,0.00}{##1}}}
\expandafter\def\csname PY@tok@ge\endcsname{\let\PY@it=\textit}
\expandafter\def\csname PY@tok@gs\endcsname{\let\PY@bf=\textbf}
\expandafter\def\csname PY@tok@gp\endcsname{\let\PY@bf=\textbf\def\PY@tc##1{\textcolor[rgb]{0.00,0.00,0.50}{##1}}}
\expandafter\def\csname PY@tok@go\endcsname{\def\PY@tc##1{\textcolor[rgb]{0.53,0.53,0.53}{##1}}}
\expandafter\def\csname PY@tok@gt\endcsname{\def\PY@tc##1{\textcolor[rgb]{0.00,0.27,0.87}{##1}}}
\expandafter\def\csname PY@tok@err\endcsname{\def\PY@bc##1{\setlength{\fboxsep}{0pt}\fcolorbox[rgb]{1.00,0.00,0.00}{1,1,1}{\strut ##1}}}
\expandafter\def\csname PY@tok@kc\endcsname{\let\PY@bf=\textbf\def\PY@tc##1{\textcolor[rgb]{0.00,0.50,0.00}{##1}}}
\expandafter\def\csname PY@tok@kd\endcsname{\let\PY@bf=\textbf\def\PY@tc##1{\textcolor[rgb]{0.00,0.50,0.00}{##1}}}
\expandafter\def\csname PY@tok@kn\endcsname{\let\PY@bf=\textbf\def\PY@tc##1{\textcolor[rgb]{0.00,0.50,0.00}{##1}}}
\expandafter\def\csname PY@tok@kr\endcsname{\let\PY@bf=\textbf\def\PY@tc##1{\textcolor[rgb]{0.00,0.50,0.00}{##1}}}
\expandafter\def\csname PY@tok@bp\endcsname{\def\PY@tc##1{\textcolor[rgb]{0.00,0.50,0.00}{##1}}}
\expandafter\def\csname PY@tok@vc\endcsname{\def\PY@tc##1{\textcolor[rgb]{0.10,0.09,0.49}{##1}}}
\expandafter\def\csname PY@tok@vg\endcsname{\def\PY@tc##1{\textcolor[rgb]{0.10,0.09,0.49}{##1}}}
\expandafter\def\csname PY@tok@vi\endcsname{\def\PY@tc##1{\textcolor[rgb]{0.10,0.09,0.49}{##1}}}
\expandafter\def\csname PY@tok@sb\endcsname{\def\PY@tc##1{\textcolor[rgb]{0.73,0.13,0.13}{##1}}}
\expandafter\def\csname PY@tok@sc\endcsname{\def\PY@tc##1{\textcolor[rgb]{0.73,0.13,0.13}{##1}}}
\expandafter\def\csname PY@tok@s2\endcsname{\def\PY@tc##1{\textcolor[rgb]{0.73,0.13,0.13}{##1}}}
\expandafter\def\csname PY@tok@sh\endcsname{\def\PY@tc##1{\textcolor[rgb]{0.73,0.13,0.13}{##1}}}
\expandafter\def\csname PY@tok@s1\endcsname{\def\PY@tc##1{\textcolor[rgb]{0.73,0.13,0.13}{##1}}}
\expandafter\def\csname PY@tok@mb\endcsname{\def\PY@tc##1{\textcolor[rgb]{0.40,0.40,0.40}{##1}}}
\expandafter\def\csname PY@tok@mf\endcsname{\def\PY@tc##1{\textcolor[rgb]{0.40,0.40,0.40}{##1}}}
\expandafter\def\csname PY@tok@mh\endcsname{\def\PY@tc##1{\textcolor[rgb]{0.40,0.40,0.40}{##1}}}
\expandafter\def\csname PY@tok@mi\endcsname{\def\PY@tc##1{\textcolor[rgb]{0.40,0.40,0.40}{##1}}}
\expandafter\def\csname PY@tok@il\endcsname{\def\PY@tc##1{\textcolor[rgb]{0.40,0.40,0.40}{##1}}}
\expandafter\def\csname PY@tok@mo\endcsname{\def\PY@tc##1{\textcolor[rgb]{0.40,0.40,0.40}{##1}}}
\expandafter\def\csname PY@tok@ch\endcsname{\let\PY@it=\textit\def\PY@tc##1{\textcolor[rgb]{0.25,0.50,0.50}{##1}}}
\expandafter\def\csname PY@tok@cm\endcsname{\let\PY@it=\textit\def\PY@tc##1{\textcolor[rgb]{0.25,0.50,0.50}{##1}}}
\expandafter\def\csname PY@tok@cpf\endcsname{\let\PY@it=\textit\def\PY@tc##1{\textcolor[rgb]{0.25,0.50,0.50}{##1}}}
\expandafter\def\csname PY@tok@c1\endcsname{\let\PY@it=\textit\def\PY@tc##1{\textcolor[rgb]{0.25,0.50,0.50}{##1}}}
\expandafter\def\csname PY@tok@cs\endcsname{\let\PY@it=\textit\def\PY@tc##1{\textcolor[rgb]{0.25,0.50,0.50}{##1}}}

\def\PYZbs{\char`\\}
\def\PYZus{\char`\_}
\def\PYZob{\char`\{}
\def\PYZcb{\char`\}}
\def\PYZca{\char`\^}
\def\PYZam{\char`\&}
\def\PYZlt{\char`\<}
\def\PYZgt{\char`\>}
\def\PYZsh{\char`\#}
\def\PYZpc{\char`\%}
\def\PYZdl{\char`\$}
\def\PYZhy{\char`\-}
\def\PYZsq{\char`\'}
\def\PYZdq{\char`\"}
\def\PYZti{\char`\~}
% for compatibility with earlier versions
\def\PYZat{@}
\def\PYZlb{[}
\def\PYZrb{]}
\makeatother


    % Exact colors from NB
    \definecolor{incolor}{rgb}{0.0, 0.0, 0.5}
    \definecolor{outcolor}{rgb}{0.545, 0.0, 0.0}



    
    % Prevent overflowing lines due to hard-to-break entities
    \sloppy 
    % Setup hyperref package
    \hypersetup{
      breaklinks=true,  % so long urls are correctly broken across lines
      colorlinks=true,
      urlcolor=urlcolor,
      linkcolor=linkcolor,
      citecolor=citecolor,
      }
    % Slightly bigger margins than the latex defaults
    
    \geometry{verbose,tmargin=1in,bmargin=1in,lmargin=1in,rmargin=1in}
    
    

    \begin{document}
    
    
    \maketitle
    
    

    
    \subsection{Trigger Word Detection}\label{trigger-word-detection}

Welcome to the final programming assignment of this specialization!

In this week's videos, you learned about applying deep learning to
speech recognition. In this assignment, you will construct a speech
dataset and implement an algorithm for trigger word detection (sometimes
also called keyword detection, or wake word detection).

\begin{itemize}
\itemsep1pt\parskip0pt\parsep0pt
\item
  Trigger word detection is the technology that allows devices like
  Amazon Alexa, Google Home, Apple Siri, and Baidu DuerOS to wake up
  upon hearing a certain word.\\
\item
  For this exercise, our trigger word will be ``Activate.'' Every time
  it hears you say ``activate,'' it will make a ``chiming'' sound.
\item
  By the end of this assignment, you will be able to record a clip of
  yourself talking, and have the algorithm trigger a chime when it
  detects you saying ``activate.''
\item
  After completing this assignment, perhaps you can also extend it to
  run on your laptop so that every time you say ``activate'' it starts
  up your favorite app, or turns on a network connected lamp in your
  house, or triggers some other event?
\end{itemize}

In this assignment you will learn to: - Structure a speech recognition
project - Synthesize and process audio recordings to create train/dev
datasets - Train a trigger word detection model and make predictions

    \subsection{Updates}\label{updates}

\paragraph{If you were working on the notebook before this
update\ldots{}}\label{if-you-were-working-on-the-notebook-before-this-update}

\begin{itemize}
\itemsep1pt\parskip0pt\parsep0pt
\item
  The current notebook is version ``1a''.
\item
  You can find your original work saved in the notebook with the
  previous version name (``v1'')
\item
  To view the file directory, go to the menu
  ``File-\textgreater{}Open'', and this will open a new tab that shows
  the file directory.
\end{itemize}

\paragraph{List of updates}\label{list-of-updates}

\begin{itemize}
\itemsep1pt\parskip0pt\parsep0pt
\item
  2.1: build the model

  \begin{itemize}
  \itemsep1pt\parskip0pt\parsep0pt
  \item
    Added sample code to show how to use the Keras layers.
  \item
    Lets student to implement the \texttt{TimeDistributed} code.
  \end{itemize}
\item
  Spelling, grammar and wording corrections.
\end{itemize}

    Let's get started! Run the following cell to load the package you are
going to use.

    \begin{Verbatim}[commandchars=\\\{\}]
{\color{incolor}In [{\color{incolor} }]:} \PY{k+kn}{import} \PY{n+nn}{numpy} \PY{k}{as} \PY{n+nn}{np}
        \PY{k+kn}{from} \PY{n+nn}{pydub} \PY{k}{import} \PY{n}{AudioSegment}
        \PY{k+kn}{import} \PY{n+nn}{random}
        \PY{k+kn}{import} \PY{n+nn}{sys}
        \PY{k+kn}{import} \PY{n+nn}{io}
        \PY{k+kn}{import} \PY{n+nn}{os}
        \PY{k+kn}{import} \PY{n+nn}{glob}
        \PY{k+kn}{import} \PY{n+nn}{IPython}
        \PY{k+kn}{from} \PY{n+nn}{td\PYZus{}utils} \PY{k}{import} \PY{o}{*}
        \PY{o}{\PYZpc{}}\PY{k}{matplotlib} inline
\end{Verbatim}

    \section{1 - Data synthesis: Creating a speech
dataset}\label{data-synthesis-creating-a-speech-dataset}

Let's start by building a dataset for your trigger word detection
algorithm. * A speech dataset should ideally be as close as possible to
the application you will want to run it on. * In this case, you'd like
to detect the word ``activate'' in working environments (library, home,
offices, open-spaces \ldots{}). * Therefore, you need to create
recordings with a mix of positive words (``activate'') and negative
words (random words other than activate) on different background sounds.
Let's see how you can create such a dataset.

\subsection{1.1 - Listening to the data}\label{listening-to-the-data}

\begin{itemize}
\itemsep1pt\parskip0pt\parsep0pt
\item
  One of your friends is helping you out on this project, and they've
  gone to libraries, cafes, restaurants, homes and offices all around
  the region to record background noises, as well as snippets of audio
  of people saying positive/negative words. This dataset includes people
  speaking in a variety of accents.
\item
  In the raw\_data directory, you can find a subset of the raw audio
  files of the positive words, negative words, and background noise. You
  will use these audio files to synthesize a dataset to train the model.

  \begin{itemize}
  \itemsep1pt\parskip0pt\parsep0pt
  \item
    The ``activate'' directory contains positive examples of people
    saying the word ``activate''.
  \item
    The ``negatives'' directory contains negative examples of people
    saying random words other than ``activate''.
  \item
    There is one word per audio recording.
  \item
    The ``backgrounds'' directory contains 10 second clips of background
    noise in different environments.
  \end{itemize}
\end{itemize}

Run the cells below to listen to some examples.

    \begin{Verbatim}[commandchars=\\\{\}]
{\color{incolor}In [{\color{incolor} }]:} \PY{n}{IPython}\PY{o}{.}\PY{n}{display}\PY{o}{.}\PY{n}{Audio}\PY{p}{(}\PY{l+s+s2}{\PYZdq{}}\PY{l+s+s2}{./raw\PYZus{}data/activates/1.wav}\PY{l+s+s2}{\PYZdq{}}\PY{p}{)}
\end{Verbatim}

    \begin{Verbatim}[commandchars=\\\{\}]
{\color{incolor}In [{\color{incolor} }]:} \PY{n}{IPython}\PY{o}{.}\PY{n}{display}\PY{o}{.}\PY{n}{Audio}\PY{p}{(}\PY{l+s+s2}{\PYZdq{}}\PY{l+s+s2}{./raw\PYZus{}data/negatives/4.wav}\PY{l+s+s2}{\PYZdq{}}\PY{p}{)}
\end{Verbatim}

    \begin{Verbatim}[commandchars=\\\{\}]
{\color{incolor}In [{\color{incolor} }]:} \PY{n}{IPython}\PY{o}{.}\PY{n}{display}\PY{o}{.}\PY{n}{Audio}\PY{p}{(}\PY{l+s+s2}{\PYZdq{}}\PY{l+s+s2}{./raw\PYZus{}data/backgrounds/1.wav}\PY{l+s+s2}{\PYZdq{}}\PY{p}{)}
\end{Verbatim}

    You will use these three types of recordings
(positives/negatives/backgrounds) to create a labeled dataset.

    \subsection{1.2 - From audio recordings to
spectrograms}\label{from-audio-recordings-to-spectrograms}

What really is an audio recording? * A microphone records little
variations in air pressure over time, and it is these little variations
in air pressure that your ear also perceives as sound. * You can think
of an audio recording is a long list of numbers measuring the little air
pressure changes detected by the microphone. * We will use audio sampled
at 44100 Hz (or 44100 Hertz). * This means the microphone gives us
44,100 numbers per second. * Thus, a 10 second audio clip is represented
by 441,000 numbers (= $10 \times 44,100$).

\paragraph{Spectrogram}\label{spectrogram}

\begin{itemize}
\itemsep1pt\parskip0pt\parsep0pt
\item
  It is quite difficult to figure out from this ``raw'' representation
  of audio whether the word ``activate'' was said.
\item
  In order to help your sequence model more easily learn to detect
  trigger words, we will compute a \emph{spectrogram} of the audio.
\item
  The spectrogram tells us how much different frequencies are present in
  an audio clip at any moment in time.
\item
  If you've ever taken an advanced class on signal processing or on
  Fourier transforms:

  \begin{itemize}
  \itemsep1pt\parskip0pt\parsep0pt
  \item
    A spectrogram is computed by sliding a window over the raw audio
    signal, and calculating the most active frequencies in each window
    using a Fourier transform.
  \item
    If you don't understand the previous sentence, don't worry about it.
  \end{itemize}
\end{itemize}

Let's look at an example.

    \begin{Verbatim}[commandchars=\\\{\}]
{\color{incolor}In [{\color{incolor} }]:} \PY{n}{IPython}\PY{o}{.}\PY{n}{display}\PY{o}{.}\PY{n}{Audio}\PY{p}{(}\PY{l+s+s2}{\PYZdq{}}\PY{l+s+s2}{audio\PYZus{}examples/example\PYZus{}train.wav}\PY{l+s+s2}{\PYZdq{}}\PY{p}{)}
\end{Verbatim}

    \begin{Verbatim}[commandchars=\\\{\}]
{\color{incolor}In [{\color{incolor} }]:} \PY{n}{x} \PY{o}{=} \PY{n}{graph\PYZus{}spectrogram}\PY{p}{(}\PY{l+s+s2}{\PYZdq{}}\PY{l+s+s2}{audio\PYZus{}examples/example\PYZus{}train.wav}\PY{l+s+s2}{\PYZdq{}}\PY{p}{)}
\end{Verbatim}

    The graph above represents how active each frequency is (y axis) over a
number of time-steps (x axis).

\textbf{Figure 1}: Spectrogram of an audio recording

\begin{itemize}
\itemsep1pt\parskip0pt\parsep0pt
\item
  The color in the spectrogram shows the degree to which different
  frequencies are present (loud) in the audio at different points in
  time.
\item
  Green means a certain frequency is more active or more present in the
  audio clip (louder).
\item
  Blue squares denote less active frequencies.
\item
  The dimension of the output spectrogram depends upon the
  hyperparameters of the spectrogram software and the length of the
  input.
\item
  In this notebook, we will be working with 10 second audio clips as the
  ``standard length'' for our training examples.

  \begin{itemize}
  \itemsep1pt\parskip0pt\parsep0pt
  \item
    The number of timesteps of the spectrogram will be 5511.
  \item
    You'll see later that the spectrogram will be the input $x$ into the
    network, and so $T_x = 5511$.
  \end{itemize}
\end{itemize}

    \begin{Verbatim}[commandchars=\\\{\}]
{\color{incolor}In [{\color{incolor} }]:} \PY{n}{\PYZus{}}\PY{p}{,} \PY{n}{data} \PY{o}{=} \PY{n}{wavfile}\PY{o}{.}\PY{n}{read}\PY{p}{(}\PY{l+s+s2}{\PYZdq{}}\PY{l+s+s2}{audio\PYZus{}examples/example\PYZus{}train.wav}\PY{l+s+s2}{\PYZdq{}}\PY{p}{)}
        \PY{n+nb}{print}\PY{p}{(}\PY{l+s+s2}{\PYZdq{}}\PY{l+s+s2}{Time steps in audio recording before spectrogram}\PY{l+s+s2}{\PYZdq{}}\PY{p}{,} \PY{n}{data}\PY{p}{[}\PY{p}{:}\PY{p}{,}\PY{l+m+mi}{0}\PY{p}{]}\PY{o}{.}\PY{n}{shape}\PY{p}{)}
        \PY{n+nb}{print}\PY{p}{(}\PY{l+s+s2}{\PYZdq{}}\PY{l+s+s2}{Time steps in input after spectrogram}\PY{l+s+s2}{\PYZdq{}}\PY{p}{,} \PY{n}{x}\PY{o}{.}\PY{n}{shape}\PY{p}{)}
\end{Verbatim}

    Now, you can define:

    \begin{Verbatim}[commandchars=\\\{\}]
{\color{incolor}In [{\color{incolor} }]:} \PY{n}{Tx} \PY{o}{=} \PY{l+m+mi}{5511} \PY{c+c1}{\PYZsh{} The number of time steps input to the model from the spectrogram}
        \PY{n}{n\PYZus{}freq} \PY{o}{=} \PY{l+m+mi}{101} \PY{c+c1}{\PYZsh{} Number of frequencies input to the model at each time step of the spectrogram}
\end{Verbatim}

    \paragraph{Dividing into
time-intervals}\label{dividing-into-time-intervals}

Note that we may divide a 10 second interval of time with different
units (steps). * Raw audio divides 10 seconds into 441,000 units. * A
spectrogram divides 10 seconds into 5,511 units. * $T_x = 5511$ * You
will use a Python module \texttt{pydub} to synthesize audio, and it
divides 10 seconds into 10,000 units. * The output of our model will
divide 10 seconds into 1,375 units. * $T_y = 1375$ * For each of the
1375 time steps, the model predicts whether someone recently finished
saying the trigger word ``activate.'' * All of these are hyperparameters
and can be changed (except the 441000, which is a function of the
microphone). * We have chosen values that are within the standard range
used for speech systems.

    \begin{Verbatim}[commandchars=\\\{\}]
{\color{incolor}In [{\color{incolor} }]:} \PY{n}{Ty} \PY{o}{=} \PY{l+m+mi}{1375} \PY{c+c1}{\PYZsh{} The number of time steps in the output of our model}
\end{Verbatim}

    \subsection{1.3 - Generating a single training
example}\label{generating-a-single-training-example}

\paragraph{Benefits of synthesizing
data}\label{benefits-of-synthesizing-data}

Because speech data is hard to acquire and label, you will synthesize
your training data using the audio clips of activates, negatives, and
backgrounds. * It is quite slow to record lots of 10 second audio clips
with random ``activates'' in it. * Instead, it is easier to record lots
of positives and negative words, and record background noise separately
(or download background noise from free online sources).

\paragraph{Process for Synthesizing an audio
clip}\label{process-for-synthesizing-an-audio-clip}

\begin{itemize}
\itemsep1pt\parskip0pt\parsep0pt
\item
  To synthesize a single training example, you will:

  \begin{itemize}
  \itemsep1pt\parskip0pt\parsep0pt
  \item
    Pick a random 10 second background audio clip
  \item
    Randomly insert 0-4 audio clips of ``activate'' into this 10sec clip
  \item
    Randomly insert 0-2 audio clips of negative words into this 10sec
    clip
  \end{itemize}
\item
  Because you had synthesized the word ``activate'' into the background
  clip, you know exactly when in the 10 second clip the ``activate''
  makes its appearance.

  \begin{itemize}
  \itemsep1pt\parskip0pt\parsep0pt
  \item
    You'll see later that this makes it easier to generate the labels
    $y^{\langle t \rangle}$ as well.
  \end{itemize}
\end{itemize}

\paragraph{Pydub}\label{pydub}

\begin{itemize}
\itemsep1pt\parskip0pt\parsep0pt
\item
  You will use the pydub package to manipulate audio.
\item
  Pydub converts raw audio files into lists of Pydub data structures.

  \begin{itemize}
  \itemsep1pt\parskip0pt\parsep0pt
  \item
    Don't worry about the details of the data structures.
  \end{itemize}
\item
  Pydub uses 1ms as the discretization interval (1ms is 1 millisecond =
  1/1000 seconds).

  \begin{itemize}
  \itemsep1pt\parskip0pt\parsep0pt
  \item
    This is why a 10 second clip is always represented using 10,000
    steps.
  \end{itemize}
\end{itemize}

    \begin{Verbatim}[commandchars=\\\{\}]
{\color{incolor}In [{\color{incolor} }]:} \PY{c+c1}{\PYZsh{} Load audio segments using pydub }
        \PY{n}{activates}\PY{p}{,} \PY{n}{negatives}\PY{p}{,} \PY{n}{backgrounds} \PY{o}{=} \PY{n}{load\PYZus{}raw\PYZus{}audio}\PY{p}{(}\PY{p}{)}
        
        \PY{n+nb}{print}\PY{p}{(}\PY{l+s+s2}{\PYZdq{}}\PY{l+s+s2}{background len should be 10,000, since it is a 10 sec clip}\PY{l+s+se}{\PYZbs{}n}\PY{l+s+s2}{\PYZdq{}} \PY{o}{+} \PY{n+nb}{str}\PY{p}{(}\PY{n+nb}{len}\PY{p}{(}\PY{n}{backgrounds}\PY{p}{[}\PY{l+m+mi}{0}\PY{p}{]}\PY{p}{)}\PY{p}{)}\PY{p}{,}\PY{l+s+s2}{\PYZdq{}}\PY{l+s+se}{\PYZbs{}n}\PY{l+s+s2}{\PYZdq{}}\PY{p}{)}
        \PY{n+nb}{print}\PY{p}{(}\PY{l+s+s2}{\PYZdq{}}\PY{l+s+s2}{activate[0] len may be around 1000, since an `activate` audio clip is usually around 1 second (but varies a lot) }\PY{l+s+se}{\PYZbs{}n}\PY{l+s+s2}{\PYZdq{}} \PY{o}{+} \PY{n+nb}{str}\PY{p}{(}\PY{n+nb}{len}\PY{p}{(}\PY{n}{activates}\PY{p}{[}\PY{l+m+mi}{0}\PY{p}{]}\PY{p}{)}\PY{p}{)}\PY{p}{,}\PY{l+s+s2}{\PYZdq{}}\PY{l+s+se}{\PYZbs{}n}\PY{l+s+s2}{\PYZdq{}}\PY{p}{)}
        \PY{n+nb}{print}\PY{p}{(}\PY{l+s+s2}{\PYZdq{}}\PY{l+s+s2}{activate[1] len: different `activate` clips can have different lengths}\PY{l+s+se}{\PYZbs{}n}\PY{l+s+s2}{\PYZdq{}} \PY{o}{+} \PY{n+nb}{str}\PY{p}{(}\PY{n+nb}{len}\PY{p}{(}\PY{n}{activates}\PY{p}{[}\PY{l+m+mi}{1}\PY{p}{]}\PY{p}{)}\PY{p}{)}\PY{p}{,}\PY{l+s+s2}{\PYZdq{}}\PY{l+s+se}{\PYZbs{}n}\PY{l+s+s2}{\PYZdq{}}\PY{p}{)}
\end{Verbatim}

    \subsubsection{Overlaying positive/negative `word' audio clips on top of
the background
audio}\label{overlaying-positivenegative-word-audio-clips-on-top-of-the-background-audio}

\begin{itemize}
\itemsep1pt\parskip0pt\parsep0pt
\item
  Given a 10 second background clip and a short audio clip containing a
  positive or negative word, you need to be able to ``add'' the word
  audio clip on top of the background audio.
\item
  You will be inserting multiple clips of positive/negative words into
  the background, and you don't want to insert an ``activate'' or a
  random word somewhere that overlaps with another clip you had
  previously added.

  \begin{itemize}
  \itemsep1pt\parskip0pt\parsep0pt
  \item
    To ensure that the `word' audio segments do not overlap when
    inserted, you will keep track of the times of previously inserted
    audio clips.
  \end{itemize}
\item
  To be clear, when you insert a 1 second ``activate'' onto a 10 second
  clip of cafe noise, \textbf{you do not end up with an 11 sec clip.}

  \begin{itemize}
  \itemsep1pt\parskip0pt\parsep0pt
  \item
    The resulting audio clip is still 10 seconds long.
  \item
    You'll see later how pydub allows you to do this.
  \end{itemize}
\end{itemize}

    \paragraph{Label the positive/negative
words}\label{label-the-positivenegative-words}

\begin{itemize}
\itemsep1pt\parskip0pt\parsep0pt
\item
  Recall that the labels $y^{\langle t \rangle}$ represent whether or
  not someone has just finished saying ``activate.''

  \begin{itemize}
  \itemsep1pt\parskip0pt\parsep0pt
  \item
    $y^{\langle t \rangle} = 1$ when that that clip has finished saying
    ``activate''.
  \item
    Given a background clip, we can initialize $y^{\langle t \rangle}=0$
    for all $t$, since the clip doesn't contain any ``activates.''
  \end{itemize}
\item
  When you insert or overlay an ``activate'' clip, you will also update
  labels for $y^{\langle t \rangle}$.

  \begin{itemize}
  \itemsep1pt\parskip0pt\parsep0pt
  \item
    Rather than updating the label of a single time step, we will update
    50 steps of the output to have target label 1.
  \item
    Recall from the lecture on trigger word detection that updating
    several consecutive time steps can make the training data more
    balanced.
  \end{itemize}
\item
  You will train a GRU (Gated Recurrent Unit) to detect when someone has
  \textbf{finished} saying ``activate''.
\end{itemize}

\subparagraph{Example}\label{example}

\begin{itemize}
\itemsep1pt\parskip0pt\parsep0pt
\item
  Suppose the synthesized ``activate'' clip ends at the 5 second mark in
  the 10 second audio - exactly halfway into the clip.
\item
  Recall that $T_y = 1375$, so timestep $687 = $ \texttt{int(1375*0.5)}
  corresponds to the moment 5 seconds into the audio clip.
\item
  Set $y^{\langle 688 \rangle} = 1$.
\item
  We will allow the GRU to detect ``activate'' anywhere within a short
  time-internal \textbf{after} this moment, so we actually \textbf{set
  50 consecutive values} of the label $y^{\langle t \rangle}$ to 1.

  \begin{itemize}
  \itemsep1pt\parskip0pt\parsep0pt
  \item
    Specifically, we have
    $y^{\langle 688 \rangle} = y^{\langle 689 \rangle} = \cdots = y^{\langle 737 \rangle} = 1$.
  \end{itemize}
\end{itemize}

\subparagraph{Synthesized data is easier to
label}\label{synthesized-data-is-easier-to-label}

\begin{itemize}
\itemsep1pt\parskip0pt\parsep0pt
\item
  This is another reason for synthesizing the training data: It's
  relatively straightforward to generate these labels
  $y^{\langle t \rangle}$ as described above.
\item
  In contrast, if you have 10sec of audio recorded on a microphone, it's
  quite time consuming for a person to listen to it and mark manually
  exactly when ``activate'' finished.
\end{itemize}

    \paragraph{Visualizing the labels}\label{visualizing-the-labels}

\begin{itemize}
\itemsep1pt\parskip0pt\parsep0pt
\item
  Here's a figure illustrating the labels $y^{\langle t \rangle}$ in a
  clip.

  \begin{itemize}
  \itemsep1pt\parskip0pt\parsep0pt
  \item
    We have inserted ``activate'', ``innocent'', activate``,''baby."
  \item
    Note that the positive labels ``1'' are associated only with the
    positive words.
  \end{itemize}
\end{itemize}

\textbf{Figure 2}

    \paragraph{Helper functions}\label{helper-functions}

To implement the training set synthesis process, you will use the
following helper functions. * All of these functions will use a 1ms
discretization interval * The 10 seconds of audio is always discretized
into 10,000 steps.

\begin{enumerate}
\def\labelenumi{\arabic{enumi}.}
\itemsep1pt\parskip0pt\parsep0pt
\item
  \texttt{get\_random\_time\_segment(segment\_ms)}

  \begin{itemize}
  \itemsep1pt\parskip0pt\parsep0pt
  \item
    Retrieves a random time segment from the background audio.
  \end{itemize}
\item
  \texttt{is\_overlapping(segment\_time, existing\_segments)}

  \begin{itemize}
  \itemsep1pt\parskip0pt\parsep0pt
  \item
    Checks if a time segment overlaps with existing segments
  \end{itemize}
\item
  \texttt{insert\_audio\_clip(background, audio\_clip, existing\_times)}

  \begin{itemize}
  \itemsep1pt\parskip0pt\parsep0pt
  \item
    Inserts an audio segment at a random time in the background audio
  \item
    Uses the functions \texttt{get\_random\_time\_segment} and
    \texttt{is\_overlapping}
  \end{itemize}
\item
  \texttt{insert\_ones(y, segment\_end\_ms)}

  \begin{itemize}
  \itemsep1pt\parskip0pt\parsep0pt
  \item
    Inserts additional 1's into the label vector y after the word
    ``activate''
  \end{itemize}
\end{enumerate}

    \paragraph{Get a random time segment}\label{get-a-random-time-segment}

\begin{itemize}
\itemsep1pt\parskip0pt\parsep0pt
\item
  The function \texttt{get\_random\_time\_segment(segment\_ms)} returns
  a random time segment onto which we can insert an audio clip of
  duration \texttt{segment\_ms}.
\item
  Please read through the code to make sure you understand what it is
  doing.
\end{itemize}

    \begin{Verbatim}[commandchars=\\\{\}]
{\color{incolor}In [{\color{incolor} }]:} \PY{k}{def} \PY{n+nf}{get\PYZus{}random\PYZus{}time\PYZus{}segment}\PY{p}{(}\PY{n}{segment\PYZus{}ms}\PY{p}{)}\PY{p}{:}
            \PY{l+s+sd}{\PYZdq{}\PYZdq{}\PYZdq{}}
        \PY{l+s+sd}{    Gets a random time segment of duration segment\PYZus{}ms in a 10,000 ms audio clip.}
        \PY{l+s+sd}{    }
        \PY{l+s+sd}{    Arguments:}
        \PY{l+s+sd}{    segment\PYZus{}ms \PYZhy{}\PYZhy{} the duration of the audio clip in ms (\PYZdq{}ms\PYZdq{} stands for \PYZdq{}milliseconds\PYZdq{})}
        \PY{l+s+sd}{    }
        \PY{l+s+sd}{    Returns:}
        \PY{l+s+sd}{    segment\PYZus{}time \PYZhy{}\PYZhy{} a tuple of (segment\PYZus{}start, segment\PYZus{}end) in ms}
        \PY{l+s+sd}{    \PYZdq{}\PYZdq{}\PYZdq{}}
            
            \PY{n}{segment\PYZus{}start} \PY{o}{=} \PY{n}{np}\PY{o}{.}\PY{n}{random}\PY{o}{.}\PY{n}{randint}\PY{p}{(}\PY{n}{low}\PY{o}{=}\PY{l+m+mi}{0}\PY{p}{,} \PY{n}{high}\PY{o}{=}\PY{l+m+mi}{10000}\PY{o}{\PYZhy{}}\PY{n}{segment\PYZus{}ms}\PY{p}{)}   \PY{c+c1}{\PYZsh{} Make sure segment doesn\PYZsq{}t run past the 10sec background }
            \PY{n}{segment\PYZus{}end} \PY{o}{=} \PY{n}{segment\PYZus{}start} \PY{o}{+} \PY{n}{segment\PYZus{}ms} \PY{o}{\PYZhy{}} \PY{l+m+mi}{1}
            
            \PY{k}{return} \PY{p}{(}\PY{n}{segment\PYZus{}start}\PY{p}{,} \PY{n}{segment\PYZus{}end}\PY{p}{)}
\end{Verbatim}

    \paragraph{Check if audio clips are
overlapping}\label{check-if-audio-clips-are-overlapping}

\begin{itemize}
\itemsep1pt\parskip0pt\parsep0pt
\item
  Suppose you have inserted audio clips at segments (1000,1800) and
  (3400,4500).

  \begin{itemize}
  \itemsep1pt\parskip0pt\parsep0pt
  \item
    The first segment starts at step 1000 and ends at step 1800.
  \item
    The second segment starts at 3400 and ends at 4500.
  \end{itemize}
\item
  If we are considering whether to insert a new audio clip at
  (3000,3600) does this overlap with one of the previously inserted
  segments?

  \begin{itemize}
  \itemsep1pt\parskip0pt\parsep0pt
  \item
    In this case, (3000,3600) and (3400,4500) overlap, so we should
    decide against inserting a clip here.
  \end{itemize}
\item
  For the purpose of this function, define (100,200) and (200,250) to be
  overlapping, since they overlap at timestep 200.
\item
  (100,199) and (200,250) are non-overlapping.
\end{itemize}

\textbf{Exercise}: * Implement
\texttt{is\_overlapping(segment\_time, existing\_segments)} to check if
a new time segment overlaps with any of the previous segments. * You
will need to carry out 2 steps:

\begin{enumerate}
\def\labelenumi{\arabic{enumi}.}
\itemsep1pt\parskip0pt\parsep0pt
\item
  Create a ``False'' flag, that you will later set to ``True'' if you
  find that there is an overlap.
\item
  Loop over the previous\_segments' start and end times. Compare these
  times to the segment's start and end times. If there is an overlap,
  set the flag defined in (1) as True.
\end{enumerate}

You can use:

\begin{Shaded}
\begin{Highlighting}[]
\KeywordTok{for} \NormalTok{....:}
        \KeywordTok{if} \NormalTok{... <= ... and ... >= ...:}
            \NormalTok{...}
\end{Highlighting}
\end{Shaded}

Hint: There is overlap if: * The new segment starts before the previous
segment ends \textbf{and} * The new segment ends after the previous
segment starts.

    \begin{Verbatim}[commandchars=\\\{\}]
{\color{incolor}In [{\color{incolor} }]:} \PY{c+c1}{\PYZsh{} GRADED FUNCTION: is\PYZus{}overlapping}
        
        \PY{k}{def} \PY{n+nf}{is\PYZus{}overlapping}\PY{p}{(}\PY{n}{segment\PYZus{}time}\PY{p}{,} \PY{n}{previous\PYZus{}segments}\PY{p}{)}\PY{p}{:}
            \PY{l+s+sd}{\PYZdq{}\PYZdq{}\PYZdq{}}
        \PY{l+s+sd}{    Checks if the time of a segment overlaps with the times of existing segments.}
        \PY{l+s+sd}{    }
        \PY{l+s+sd}{    Arguments:}
        \PY{l+s+sd}{    segment\PYZus{}time \PYZhy{}\PYZhy{} a tuple of (segment\PYZus{}start, segment\PYZus{}end) for the new segment}
        \PY{l+s+sd}{    previous\PYZus{}segments \PYZhy{}\PYZhy{} a list of tuples of (segment\PYZus{}start, segment\PYZus{}end) for the existing segments}
        \PY{l+s+sd}{    }
        \PY{l+s+sd}{    Returns:}
        \PY{l+s+sd}{    True if the time segment overlaps with any of the existing segments, False otherwise}
        \PY{l+s+sd}{    \PYZdq{}\PYZdq{}\PYZdq{}}
            
            \PY{n}{segment\PYZus{}start}\PY{p}{,} \PY{n}{segment\PYZus{}end} \PY{o}{=} \PY{n}{segment\PYZus{}time}
            
            \PY{c+c1}{\PYZsh{}\PYZsh{}\PYZsh{} START CODE HERE \PYZsh{}\PYZsh{}\PYZsh{} (≈ 4 lines)}
            \PY{c+c1}{\PYZsh{} Step 1: Initialize overlap as a \PYZdq{}False\PYZdq{} flag. (≈ 1 line)}
            \PY{n}{overlap} \PY{o}{=} \PY{k+kc}{None}
            
            \PY{c+c1}{\PYZsh{} Step 2: loop over the previous\PYZus{}segments start and end times.}
            \PY{c+c1}{\PYZsh{} Compare start/end times and set the flag to True if there is an overlap (≈ 3 lines)}
            \PY{k}{for} \PY{n}{previous\PYZus{}start}\PY{p}{,} \PY{n}{previous\PYZus{}end} \PY{o+ow}{in} \PY{n}{previous\PYZus{}segments}\PY{p}{:}
                \PY{k}{if} \PY{k+kc}{None}\PY{p}{:}
                    \PY{n}{overlap} \PY{o}{=} \PY{k+kc}{None}
            \PY{c+c1}{\PYZsh{}\PYZsh{}\PYZsh{} END CODE HERE \PYZsh{}\PYZsh{}\PYZsh{}}
        
            \PY{k}{return} \PY{n}{overlap}
\end{Verbatim}

    \begin{Verbatim}[commandchars=\\\{\}]
{\color{incolor}In [{\color{incolor} }]:} \PY{n}{overlap1} \PY{o}{=} \PY{n}{is\PYZus{}overlapping}\PY{p}{(}\PY{p}{(}\PY{l+m+mi}{950}\PY{p}{,} \PY{l+m+mi}{1430}\PY{p}{)}\PY{p}{,} \PY{p}{[}\PY{p}{(}\PY{l+m+mi}{2000}\PY{p}{,} \PY{l+m+mi}{2550}\PY{p}{)}\PY{p}{,} \PY{p}{(}\PY{l+m+mi}{260}\PY{p}{,} \PY{l+m+mi}{949}\PY{p}{)}\PY{p}{]}\PY{p}{)}
        \PY{n}{overlap2} \PY{o}{=} \PY{n}{is\PYZus{}overlapping}\PY{p}{(}\PY{p}{(}\PY{l+m+mi}{2305}\PY{p}{,} \PY{l+m+mi}{2950}\PY{p}{)}\PY{p}{,} \PY{p}{[}\PY{p}{(}\PY{l+m+mi}{824}\PY{p}{,} \PY{l+m+mi}{1532}\PY{p}{)}\PY{p}{,} \PY{p}{(}\PY{l+m+mi}{1900}\PY{p}{,} \PY{l+m+mi}{2305}\PY{p}{)}\PY{p}{,} \PY{p}{(}\PY{l+m+mi}{3424}\PY{p}{,} \PY{l+m+mi}{3656}\PY{p}{)}\PY{p}{]}\PY{p}{)}
        \PY{n+nb}{print}\PY{p}{(}\PY{l+s+s2}{\PYZdq{}}\PY{l+s+s2}{Overlap 1 = }\PY{l+s+s2}{\PYZdq{}}\PY{p}{,} \PY{n}{overlap1}\PY{p}{)}
        \PY{n+nb}{print}\PY{p}{(}\PY{l+s+s2}{\PYZdq{}}\PY{l+s+s2}{Overlap 2 = }\PY{l+s+s2}{\PYZdq{}}\PY{p}{,} \PY{n}{overlap2}\PY{p}{)}
\end{Verbatim}

    \textbf{Expected Output}:

\textbf{Overlap 1}

False

\textbf{Overlap 2}

True

    \paragraph{Insert audio clip}\label{insert-audio-clip}

\begin{itemize}
\itemsep1pt\parskip0pt\parsep0pt
\item
  Let's use the previous helper functions to insert a new audio clip
  onto the 10 second background at a random time.
\item
  We will ensure that any newly inserted segment doesn't overlap with
  previously inserted segments.
\end{itemize}

\textbf{Exercise}: * Implement \texttt{insert\_audio\_clip()} to overlay
an audio clip onto the background 10sec clip. * You implement 4 steps:

\begin{enumerate}
\def\labelenumi{\arabic{enumi}.}
\itemsep1pt\parskip0pt\parsep0pt
\item
  Get the length of the audio clip that is to be inserted.

  \begin{itemize}
  \itemsep1pt\parskip0pt\parsep0pt
  \item
    Get a random time segment whose duration equals the duration of the
    audio clip that is to be inserted.
  \end{itemize}
\item
  Make sure that the time segment does not overlap with any of the
  previous time segments.

  \begin{itemize}
  \itemsep1pt\parskip0pt\parsep0pt
  \item
    If it is overlapping, then go back to step 1 and pick a new time
    segment.
  \end{itemize}
\item
  Append the new time segment to the list of existing time segments

  \begin{itemize}
  \itemsep1pt\parskip0pt\parsep0pt
  \item
    This keeps track of all the segments you've inserted.\\
  \end{itemize}
\item
  Overlay the audio clip over the background using pydub. We have
  implemented this for you.
\end{enumerate}

    \begin{Verbatim}[commandchars=\\\{\}]
{\color{incolor}In [{\color{incolor} }]:} \PY{c+c1}{\PYZsh{} GRADED FUNCTION: insert\PYZus{}audio\PYZus{}clip}
        
        \PY{k}{def} \PY{n+nf}{insert\PYZus{}audio\PYZus{}clip}\PY{p}{(}\PY{n}{background}\PY{p}{,} \PY{n}{audio\PYZus{}clip}\PY{p}{,} \PY{n}{previous\PYZus{}segments}\PY{p}{)}\PY{p}{:}
            \PY{l+s+sd}{\PYZdq{}\PYZdq{}\PYZdq{}}
        \PY{l+s+sd}{    Insert a new audio segment over the background noise at a random time step, ensuring that the }
        \PY{l+s+sd}{    audio segment does not overlap with existing segments.}
        \PY{l+s+sd}{    }
        \PY{l+s+sd}{    Arguments:}
        \PY{l+s+sd}{    background \PYZhy{}\PYZhy{} a 10 second background audio recording.  }
        \PY{l+s+sd}{    audio\PYZus{}clip \PYZhy{}\PYZhy{} the audio clip to be inserted/overlaid. }
        \PY{l+s+sd}{    previous\PYZus{}segments \PYZhy{}\PYZhy{} times where audio segments have already been placed}
        \PY{l+s+sd}{    }
        \PY{l+s+sd}{    Returns:}
        \PY{l+s+sd}{    new\PYZus{}background \PYZhy{}\PYZhy{} the updated background audio}
        \PY{l+s+sd}{    \PYZdq{}\PYZdq{}\PYZdq{}}
            
            \PY{c+c1}{\PYZsh{} Get the duration of the audio clip in ms}
            \PY{n}{segment\PYZus{}ms} \PY{o}{=} \PY{n+nb}{len}\PY{p}{(}\PY{n}{audio\PYZus{}clip}\PY{p}{)}
            
            \PY{c+c1}{\PYZsh{}\PYZsh{}\PYZsh{} START CODE HERE \PYZsh{}\PYZsh{}\PYZsh{} }
            \PY{c+c1}{\PYZsh{} Step 1: Use one of the helper functions to pick a random time segment onto which to insert }
            \PY{c+c1}{\PYZsh{} the new audio clip. (≈ 1 line)}
            \PY{n}{segment\PYZus{}time} \PY{o}{=} \PY{k+kc}{None}
            
            \PY{c+c1}{\PYZsh{} Step 2: Check if the new segment\PYZus{}time overlaps with one of the previous\PYZus{}segments. If so, keep }
            \PY{c+c1}{\PYZsh{} picking new segment\PYZus{}time at random until it doesn\PYZsq{}t overlap. (≈ 2 lines)}
            \PY{k}{while} \PY{k+kc}{None}\PY{p}{:}
                \PY{n}{segment\PYZus{}time} \PY{o}{=} \PY{k+kc}{None}
        
            \PY{c+c1}{\PYZsh{} Step 3: Append the new segment\PYZus{}time to the list of previous\PYZus{}segments (≈ 1 line)}
            \PY{k+kc}{None}
            \PY{c+c1}{\PYZsh{}\PYZsh{}\PYZsh{} END CODE HERE \PYZsh{}\PYZsh{}\PYZsh{}}
            
            \PY{c+c1}{\PYZsh{} Step 4: Superpose audio segment and background}
            \PY{n}{new\PYZus{}background} \PY{o}{=} \PY{n}{background}\PY{o}{.}\PY{n}{overlay}\PY{p}{(}\PY{n}{audio\PYZus{}clip}\PY{p}{,} \PY{n}{position} \PY{o}{=} \PY{n}{segment\PYZus{}time}\PY{p}{[}\PY{l+m+mi}{0}\PY{p}{]}\PY{p}{)}
            
            \PY{k}{return} \PY{n}{new\PYZus{}background}\PY{p}{,} \PY{n}{segment\PYZus{}time}
\end{Verbatim}

    \begin{Verbatim}[commandchars=\\\{\}]
{\color{incolor}In [{\color{incolor} }]:} \PY{n}{np}\PY{o}{.}\PY{n}{random}\PY{o}{.}\PY{n}{seed}\PY{p}{(}\PY{l+m+mi}{5}\PY{p}{)}
        \PY{n}{audio\PYZus{}clip}\PY{p}{,} \PY{n}{segment\PYZus{}time} \PY{o}{=} \PY{n}{insert\PYZus{}audio\PYZus{}clip}\PY{p}{(}\PY{n}{backgrounds}\PY{p}{[}\PY{l+m+mi}{0}\PY{p}{]}\PY{p}{,} \PY{n}{activates}\PY{p}{[}\PY{l+m+mi}{0}\PY{p}{]}\PY{p}{,} \PY{p}{[}\PY{p}{(}\PY{l+m+mi}{3790}\PY{p}{,} \PY{l+m+mi}{4400}\PY{p}{)}\PY{p}{]}\PY{p}{)}
        \PY{n}{audio\PYZus{}clip}\PY{o}{.}\PY{n}{export}\PY{p}{(}\PY{l+s+s2}{\PYZdq{}}\PY{l+s+s2}{insert\PYZus{}test.wav}\PY{l+s+s2}{\PYZdq{}}\PY{p}{,} \PY{n+nb}{format}\PY{o}{=}\PY{l+s+s2}{\PYZdq{}}\PY{l+s+s2}{wav}\PY{l+s+s2}{\PYZdq{}}\PY{p}{)}
        \PY{n+nb}{print}\PY{p}{(}\PY{l+s+s2}{\PYZdq{}}\PY{l+s+s2}{Segment Time: }\PY{l+s+s2}{\PYZdq{}}\PY{p}{,} \PY{n}{segment\PYZus{}time}\PY{p}{)}
        \PY{n}{IPython}\PY{o}{.}\PY{n}{display}\PY{o}{.}\PY{n}{Audio}\PY{p}{(}\PY{l+s+s2}{\PYZdq{}}\PY{l+s+s2}{insert\PYZus{}test.wav}\PY{l+s+s2}{\PYZdq{}}\PY{p}{)}
\end{Verbatim}

    \textbf{Expected Output}

\textbf{Segment Time}

(2254, 3169)

    \begin{Verbatim}[commandchars=\\\{\}]
{\color{incolor}In [{\color{incolor} }]:} \PY{c+c1}{\PYZsh{} Expected audio}
        \PY{n}{IPython}\PY{o}{.}\PY{n}{display}\PY{o}{.}\PY{n}{Audio}\PY{p}{(}\PY{l+s+s2}{\PYZdq{}}\PY{l+s+s2}{audio\PYZus{}examples/insert\PYZus{}reference.wav}\PY{l+s+s2}{\PYZdq{}}\PY{p}{)}
\end{Verbatim}

    \paragraph{Insert ones for the labels of the positive
target}\label{insert-ones-for-the-labels-of-the-positive-target}

\begin{itemize}
\itemsep1pt\parskip0pt\parsep0pt
\item
  Implement code to update the labels $y^{\langle t \rangle}$, assuming
  you just inserted an ``activate'' audio clip.
\item
  In the code below, \texttt{y} is a \texttt{(1,1375)} dimensional
  vector, since $T_y = 1375$.
\item
  If the ``activate'' audio clip ends at time step $t$, then set
  $y^{\langle t+1 \rangle} = 1$ and also set the next 49 additional
  consecutive values to 1.

  \begin{itemize}
  \itemsep1pt\parskip0pt\parsep0pt
  \item
    Notice that if the target word appears near the end of the entire
    audio clip, there may not be 50 additional time steps to set to 1.
  \item
    Make sure you don't run off the end of the array and try to update
    \texttt{y{[}0{]}{[}1375{]}}, since the valid indices are
    \texttt{y{[}0{]}{[}0{]}} through \texttt{y{[}0{]}{[}1374{]}} because
    $T_y = 1375$.
  \item
    So if ``activate'' ends at step 1370, you would get only set
    \texttt{y{[}0{]}{[}1371{]} = y{[}0{]}{[}1372{]} = y{[}0{]}{[}1373{]} = y{[}0{]}{[}1374{]} = 1}
  \end{itemize}
\end{itemize}

\textbf{Exercise}: Implement \texttt{insert\_ones()}. * You can use a
for loop. * If you want to use Python's array slicing operations, you
can do so as well. * If a segment ends at \texttt{segment\_end\_ms}
(using a 10000 step discretization), * To convert it to the indexing for
the outputs $y$ (using a $1375$ step discretization), we will use this
formula:

\begin{verbatim}
    segment_end_y = int(segment_end_ms * Ty / 10000.0)
\end{verbatim}

    \begin{Verbatim}[commandchars=\\\{\}]
{\color{incolor}In [{\color{incolor} }]:} \PY{c+c1}{\PYZsh{} GRADED FUNCTION: insert\PYZus{}ones}
        
        \PY{k}{def} \PY{n+nf}{insert\PYZus{}ones}\PY{p}{(}\PY{n}{y}\PY{p}{,} \PY{n}{segment\PYZus{}end\PYZus{}ms}\PY{p}{)}\PY{p}{:}
            \PY{l+s+sd}{\PYZdq{}\PYZdq{}\PYZdq{}}
        \PY{l+s+sd}{    Update the label vector y. The labels of the 50 output steps strictly after the end of the segment }
        \PY{l+s+sd}{    should be set to 1. By strictly we mean that the label of segment\PYZus{}end\PYZus{}y should be 0 while, the}
        \PY{l+s+sd}{    50 following labels should be ones.}
        \PY{l+s+sd}{    }
        \PY{l+s+sd}{    }
        \PY{l+s+sd}{    Arguments:}
        \PY{l+s+sd}{    y \PYZhy{}\PYZhy{} numpy array of shape (1, Ty), the labels of the training example}
        \PY{l+s+sd}{    segment\PYZus{}end\PYZus{}ms \PYZhy{}\PYZhy{} the end time of the segment in ms}
        \PY{l+s+sd}{    }
        \PY{l+s+sd}{    Returns:}
        \PY{l+s+sd}{    y \PYZhy{}\PYZhy{} updated labels}
        \PY{l+s+sd}{    \PYZdq{}\PYZdq{}\PYZdq{}}
            
            \PY{c+c1}{\PYZsh{} duration of the background (in terms of spectrogram time\PYZhy{}steps)}
            \PY{n}{segment\PYZus{}end\PYZus{}y} \PY{o}{=} \PY{n+nb}{int}\PY{p}{(}\PY{n}{segment\PYZus{}end\PYZus{}ms} \PY{o}{*} \PY{n}{Ty} \PY{o}{/} \PY{l+m+mf}{10000.0}\PY{p}{)}
            
            \PY{c+c1}{\PYZsh{} Add 1 to the correct index in the background label (y)}
            \PY{c+c1}{\PYZsh{}\PYZsh{}\PYZsh{} START CODE HERE \PYZsh{}\PYZsh{}\PYZsh{} (≈ 3 lines)}
            \PY{k}{for} \PY{n}{i} \PY{o+ow}{in} \PY{n+nb}{range}\PY{p}{(}\PY{k+kc}{None}\PY{p}{,} \PY{k+kc}{None}\PY{p}{)}\PY{p}{:}
                \PY{k}{if} \PY{k+kc}{None} \PY{o}{\PYZlt{}} \PY{k+kc}{None}\PY{p}{:}
                    \PY{n}{y}\PY{p}{[}\PY{l+m+mi}{0}\PY{p}{,} \PY{n}{i}\PY{p}{]} \PY{o}{=} \PY{k+kc}{None}
            \PY{c+c1}{\PYZsh{}\PYZsh{}\PYZsh{} END CODE HERE \PYZsh{}\PYZsh{}\PYZsh{}}
            
            \PY{k}{return} \PY{n}{y}
\end{Verbatim}

    \begin{Verbatim}[commandchars=\\\{\}]
{\color{incolor}In [{\color{incolor} }]:} \PY{n}{arr1} \PY{o}{=} \PY{n}{insert\PYZus{}ones}\PY{p}{(}\PY{n}{np}\PY{o}{.}\PY{n}{zeros}\PY{p}{(}\PY{p}{(}\PY{l+m+mi}{1}\PY{p}{,} \PY{n}{Ty}\PY{p}{)}\PY{p}{)}\PY{p}{,} \PY{l+m+mi}{9700}\PY{p}{)}
        \PY{n}{plt}\PY{o}{.}\PY{n}{plot}\PY{p}{(}\PY{n}{insert\PYZus{}ones}\PY{p}{(}\PY{n}{arr1}\PY{p}{,} \PY{l+m+mi}{4251}\PY{p}{)}\PY{p}{[}\PY{l+m+mi}{0}\PY{p}{,}\PY{p}{:}\PY{p}{]}\PY{p}{)}
        \PY{n+nb}{print}\PY{p}{(}\PY{l+s+s2}{\PYZdq{}}\PY{l+s+s2}{sanity checks:}\PY{l+s+s2}{\PYZdq{}}\PY{p}{,} \PY{n}{arr1}\PY{p}{[}\PY{l+m+mi}{0}\PY{p}{]}\PY{p}{[}\PY{l+m+mi}{1333}\PY{p}{]}\PY{p}{,} \PY{n}{arr1}\PY{p}{[}\PY{l+m+mi}{0}\PY{p}{]}\PY{p}{[}\PY{l+m+mi}{634}\PY{p}{]}\PY{p}{,} \PY{n}{arr1}\PY{p}{[}\PY{l+m+mi}{0}\PY{p}{]}\PY{p}{[}\PY{l+m+mi}{635}\PY{p}{]}\PY{p}{)}
\end{Verbatim}

    \textbf{Expected Output}

\textbf{sanity checks}:

0.0 1.0 0.0

    \paragraph{Creating a training
example}\label{creating-a-training-example}

Finally, you can use \texttt{insert\_audio\_clip} and
\texttt{insert\_ones} to create a new training example.

\textbf{Exercise}: Implement \texttt{create\_training\_example()}. You
will need to carry out the following steps:

\begin{enumerate}
\def\labelenumi{\arabic{enumi}.}
\itemsep1pt\parskip0pt\parsep0pt
\item
  Initialize the label vector $y$ as a numpy array of zeros and shape
  $(1, T_y)$.
\item
  Initialize the set of existing segments to an empty list.
\item
  Randomly select 0 to 4 ``activate'' audio clips, and insert them onto
  the 10 second clip. Also insert labels at the correct position in the
  label vector $y$.
\item
  Randomly select 0 to 2 negative audio clips, and insert them into the
  10 second clip.
\end{enumerate}

    \begin{Verbatim}[commandchars=\\\{\}]
{\color{incolor}In [{\color{incolor} }]:} \PY{c+c1}{\PYZsh{} GRADED FUNCTION: create\PYZus{}training\PYZus{}example}
        
        \PY{k}{def} \PY{n+nf}{create\PYZus{}training\PYZus{}example}\PY{p}{(}\PY{n}{background}\PY{p}{,} \PY{n}{activates}\PY{p}{,} \PY{n}{negatives}\PY{p}{)}\PY{p}{:}
            \PY{l+s+sd}{\PYZdq{}\PYZdq{}\PYZdq{}}
        \PY{l+s+sd}{    Creates a training example with a given background, activates, and negatives.}
        \PY{l+s+sd}{    }
        \PY{l+s+sd}{    Arguments:}
        \PY{l+s+sd}{    background \PYZhy{}\PYZhy{} a 10 second background audio recording}
        \PY{l+s+sd}{    activates \PYZhy{}\PYZhy{} a list of audio segments of the word \PYZdq{}activate\PYZdq{}}
        \PY{l+s+sd}{    negatives \PYZhy{}\PYZhy{} a list of audio segments of random words that are not \PYZdq{}activate\PYZdq{}}
        \PY{l+s+sd}{    }
        \PY{l+s+sd}{    Returns:}
        \PY{l+s+sd}{    x \PYZhy{}\PYZhy{} the spectrogram of the training example}
        \PY{l+s+sd}{    y \PYZhy{}\PYZhy{} the label at each time step of the spectrogram}
        \PY{l+s+sd}{    \PYZdq{}\PYZdq{}\PYZdq{}}
            
            \PY{c+c1}{\PYZsh{} Set the random seed}
            \PY{n}{np}\PY{o}{.}\PY{n}{random}\PY{o}{.}\PY{n}{seed}\PY{p}{(}\PY{l+m+mi}{18}\PY{p}{)}
            
            \PY{c+c1}{\PYZsh{} Make background quieter}
            \PY{n}{background} \PY{o}{=} \PY{n}{background} \PY{o}{\PYZhy{}} \PY{l+m+mi}{20}
        
            \PY{c+c1}{\PYZsh{}\PYZsh{}\PYZsh{} START CODE HERE \PYZsh{}\PYZsh{}\PYZsh{}}
            \PY{c+c1}{\PYZsh{} Step 1: Initialize y (label vector) of zeros (≈ 1 line)}
            \PY{n}{y} \PY{o}{=} \PY{k+kc}{None}
        
            \PY{c+c1}{\PYZsh{} Step 2: Initialize segment times as an empty list (≈ 1 line)}
            \PY{n}{previous\PYZus{}segments} \PY{o}{=} \PY{k+kc}{None}
            \PY{c+c1}{\PYZsh{}\PYZsh{}\PYZsh{} END CODE HERE \PYZsh{}\PYZsh{}\PYZsh{}}
            
            \PY{c+c1}{\PYZsh{} Select 0\PYZhy{}4 random \PYZdq{}activate\PYZdq{} audio clips from the entire list of \PYZdq{}activates\PYZdq{} recordings}
            \PY{n}{number\PYZus{}of\PYZus{}activates} \PY{o}{=} \PY{n}{np}\PY{o}{.}\PY{n}{random}\PY{o}{.}\PY{n}{randint}\PY{p}{(}\PY{l+m+mi}{0}\PY{p}{,} \PY{l+m+mi}{5}\PY{p}{)}
            \PY{n}{random\PYZus{}indices} \PY{o}{=} \PY{n}{np}\PY{o}{.}\PY{n}{random}\PY{o}{.}\PY{n}{randint}\PY{p}{(}\PY{n+nb}{len}\PY{p}{(}\PY{n}{activates}\PY{p}{)}\PY{p}{,} \PY{n}{size}\PY{o}{=}\PY{n}{number\PYZus{}of\PYZus{}activates}\PY{p}{)}
            \PY{n}{random\PYZus{}activates} \PY{o}{=} \PY{p}{[}\PY{n}{activates}\PY{p}{[}\PY{n}{i}\PY{p}{]} \PY{k}{for} \PY{n}{i} \PY{o+ow}{in} \PY{n}{random\PYZus{}indices}\PY{p}{]}
            
            \PY{c+c1}{\PYZsh{}\PYZsh{}\PYZsh{} START CODE HERE \PYZsh{}\PYZsh{}\PYZsh{} (≈ 3 lines)}
            \PY{c+c1}{\PYZsh{} Step 3: Loop over randomly selected \PYZdq{}activate\PYZdq{} clips and insert in background}
            \PY{k}{for} \PY{n}{random\PYZus{}activate} \PY{o+ow}{in} \PY{n}{random\PYZus{}activates}\PY{p}{:}
                \PY{c+c1}{\PYZsh{} Insert the audio clip on the background}
                \PY{n}{background}\PY{p}{,} \PY{n}{segment\PYZus{}time} \PY{o}{=} \PY{k+kc}{None}
                \PY{c+c1}{\PYZsh{} Retrieve segment\PYZus{}start and segment\PYZus{}end from segment\PYZus{}time}
                \PY{n}{segment\PYZus{}start}\PY{p}{,} \PY{n}{segment\PYZus{}end} \PY{o}{=} \PY{k+kc}{None}
                \PY{c+c1}{\PYZsh{} Insert labels in \PYZdq{}y\PYZdq{}}
                \PY{n}{y} \PY{o}{=} \PY{k+kc}{None}
            \PY{c+c1}{\PYZsh{}\PYZsh{}\PYZsh{} END CODE HERE \PYZsh{}\PYZsh{}\PYZsh{}}
        
            \PY{c+c1}{\PYZsh{} Select 0\PYZhy{}2 random negatives audio recordings from the entire list of \PYZdq{}negatives\PYZdq{} recordings}
            \PY{n}{number\PYZus{}of\PYZus{}negatives} \PY{o}{=} \PY{n}{np}\PY{o}{.}\PY{n}{random}\PY{o}{.}\PY{n}{randint}\PY{p}{(}\PY{l+m+mi}{0}\PY{p}{,} \PY{l+m+mi}{3}\PY{p}{)}
            \PY{n}{random\PYZus{}indices} \PY{o}{=} \PY{n}{np}\PY{o}{.}\PY{n}{random}\PY{o}{.}\PY{n}{randint}\PY{p}{(}\PY{n+nb}{len}\PY{p}{(}\PY{n}{negatives}\PY{p}{)}\PY{p}{,} \PY{n}{size}\PY{o}{=}\PY{n}{number\PYZus{}of\PYZus{}negatives}\PY{p}{)}
            \PY{n}{random\PYZus{}negatives} \PY{o}{=} \PY{p}{[}\PY{n}{negatives}\PY{p}{[}\PY{n}{i}\PY{p}{]} \PY{k}{for} \PY{n}{i} \PY{o+ow}{in} \PY{n}{random\PYZus{}indices}\PY{p}{]}
        
            \PY{c+c1}{\PYZsh{}\PYZsh{}\PYZsh{} START CODE HERE \PYZsh{}\PYZsh{}\PYZsh{} (≈ 2 lines)}
            \PY{c+c1}{\PYZsh{} Step 4: Loop over randomly selected negative clips and insert in background}
            \PY{k}{for} \PY{n}{random\PYZus{}negative} \PY{o+ow}{in} \PY{n}{random\PYZus{}negatives}\PY{p}{:}
                \PY{c+c1}{\PYZsh{} Insert the audio clip on the background }
                \PY{n}{background}\PY{p}{,} \PY{n}{\PYZus{}} \PY{o}{=} \PY{k+kc}{None}
            \PY{c+c1}{\PYZsh{}\PYZsh{}\PYZsh{} END CODE HERE \PYZsh{}\PYZsh{}\PYZsh{}}
            
            \PY{c+c1}{\PYZsh{} Standardize the volume of the audio clip }
            \PY{n}{background} \PY{o}{=} \PY{n}{match\PYZus{}target\PYZus{}amplitude}\PY{p}{(}\PY{n}{background}\PY{p}{,} \PY{o}{\PYZhy{}}\PY{l+m+mf}{20.0}\PY{p}{)}
        
            \PY{c+c1}{\PYZsh{} Export new training example }
            \PY{n}{file\PYZus{}handle} \PY{o}{=} \PY{n}{background}\PY{o}{.}\PY{n}{export}\PY{p}{(}\PY{l+s+s2}{\PYZdq{}}\PY{l+s+s2}{train}\PY{l+s+s2}{\PYZdq{}} \PY{o}{+} \PY{l+s+s2}{\PYZdq{}}\PY{l+s+s2}{.wav}\PY{l+s+s2}{\PYZdq{}}\PY{p}{,} \PY{n+nb}{format}\PY{o}{=}\PY{l+s+s2}{\PYZdq{}}\PY{l+s+s2}{wav}\PY{l+s+s2}{\PYZdq{}}\PY{p}{)}
            \PY{n+nb}{print}\PY{p}{(}\PY{l+s+s2}{\PYZdq{}}\PY{l+s+s2}{File (train.wav) was saved in your directory.}\PY{l+s+s2}{\PYZdq{}}\PY{p}{)}
            
            \PY{c+c1}{\PYZsh{} Get and plot spectrogram of the new recording (background with superposition of positive and negatives)}
            \PY{n}{x} \PY{o}{=} \PY{n}{graph\PYZus{}spectrogram}\PY{p}{(}\PY{l+s+s2}{\PYZdq{}}\PY{l+s+s2}{train.wav}\PY{l+s+s2}{\PYZdq{}}\PY{p}{)}
            
            \PY{k}{return} \PY{n}{x}\PY{p}{,} \PY{n}{y}
\end{Verbatim}

    \begin{Verbatim}[commandchars=\\\{\}]
{\color{incolor}In [{\color{incolor} }]:} \PY{n}{x}\PY{p}{,} \PY{n}{y} \PY{o}{=} \PY{n}{create\PYZus{}training\PYZus{}example}\PY{p}{(}\PY{n}{backgrounds}\PY{p}{[}\PY{l+m+mi}{0}\PY{p}{]}\PY{p}{,} \PY{n}{activates}\PY{p}{,} \PY{n}{negatives}\PY{p}{)}
\end{Verbatim}

    \textbf{Expected Output} 

    Now you can listen to the training example you created and compare it to
the spectrogram generated above.

    \begin{Verbatim}[commandchars=\\\{\}]
{\color{incolor}In [{\color{incolor} }]:} \PY{n}{IPython}\PY{o}{.}\PY{n}{display}\PY{o}{.}\PY{n}{Audio}\PY{p}{(}\PY{l+s+s2}{\PYZdq{}}\PY{l+s+s2}{train.wav}\PY{l+s+s2}{\PYZdq{}}\PY{p}{)}
\end{Verbatim}

    \textbf{Expected Output}

    \begin{Verbatim}[commandchars=\\\{\}]
{\color{incolor}In [{\color{incolor} }]:} \PY{n}{IPython}\PY{o}{.}\PY{n}{display}\PY{o}{.}\PY{n}{Audio}\PY{p}{(}\PY{l+s+s2}{\PYZdq{}}\PY{l+s+s2}{audio\PYZus{}examples/train\PYZus{}reference.wav}\PY{l+s+s2}{\PYZdq{}}\PY{p}{)}
\end{Verbatim}

    Finally, you can plot the associated labels for the generated training
example.

    \begin{Verbatim}[commandchars=\\\{\}]
{\color{incolor}In [{\color{incolor} }]:} \PY{n}{plt}\PY{o}{.}\PY{n}{plot}\PY{p}{(}\PY{n}{y}\PY{p}{[}\PY{l+m+mi}{0}\PY{p}{]}\PY{p}{)}
\end{Verbatim}

    \textbf{Expected Output} 

    \subsection{1.4 - Full training set}\label{full-training-set}

\begin{itemize}
\itemsep1pt\parskip0pt\parsep0pt
\item
  You've now implemented the code needed to generate a single training
  example.
\item
  We used this process to generate a large training set.
\item
  To save time, we've already generated a set of training examples.
\end{itemize}

    \begin{Verbatim}[commandchars=\\\{\}]
{\color{incolor}In [{\color{incolor} }]:} \PY{c+c1}{\PYZsh{} Load preprocessed training examples}
        \PY{n}{X} \PY{o}{=} \PY{n}{np}\PY{o}{.}\PY{n}{load}\PY{p}{(}\PY{l+s+s2}{\PYZdq{}}\PY{l+s+s2}{./XY\PYZus{}train/X.npy}\PY{l+s+s2}{\PYZdq{}}\PY{p}{)}
        \PY{n}{Y} \PY{o}{=} \PY{n}{np}\PY{o}{.}\PY{n}{load}\PY{p}{(}\PY{l+s+s2}{\PYZdq{}}\PY{l+s+s2}{./XY\PYZus{}train/Y.npy}\PY{l+s+s2}{\PYZdq{}}\PY{p}{)}
\end{Verbatim}

    \subsection{1.5 - Development set}\label{development-set}

\begin{itemize}
\itemsep1pt\parskip0pt\parsep0pt
\item
  To test our model, we recorded a development set of 25 examples.
\item
  While our training data is synthesized, we want to create a
  development set using the same distribution as the real inputs.
\item
  Thus, we recorded 25 10-second audio clips of people saying
  ``activate'' and other random words, and labeled them by hand.
\item
  This follows the principle described in Course 3 ``Structuring Machine
  Learning Projects'' that we should create the dev set to be as similar
  as possible to the test set distribution

  \begin{itemize}
  \itemsep1pt\parskip0pt\parsep0pt
  \item
    This is why our \textbf{dev set uses real audio} rather than
    synthesized audio.
  \end{itemize}
\end{itemize}

    \begin{Verbatim}[commandchars=\\\{\}]
{\color{incolor}In [{\color{incolor} }]:} \PY{c+c1}{\PYZsh{} Load preprocessed dev set examples}
        \PY{n}{X\PYZus{}dev} \PY{o}{=} \PY{n}{np}\PY{o}{.}\PY{n}{load}\PY{p}{(}\PY{l+s+s2}{\PYZdq{}}\PY{l+s+s2}{./XY\PYZus{}dev/X\PYZus{}dev.npy}\PY{l+s+s2}{\PYZdq{}}\PY{p}{)}
        \PY{n}{Y\PYZus{}dev} \PY{o}{=} \PY{n}{np}\PY{o}{.}\PY{n}{load}\PY{p}{(}\PY{l+s+s2}{\PYZdq{}}\PY{l+s+s2}{./XY\PYZus{}dev/Y\PYZus{}dev.npy}\PY{l+s+s2}{\PYZdq{}}\PY{p}{)}
\end{Verbatim}

    \section{2 - Model}\label{model}

\begin{itemize}
\itemsep1pt\parskip0pt\parsep0pt
\item
  Now that you've built a dataset, let's write and train a trigger word
  detection model!
\item
  The model will use 1-D convolutional layers, GRU layers, and dense
  layers.
\item
  Let's load the packages that will allow you to use these layers in
  Keras. This might take a minute to load.
\end{itemize}

    \begin{Verbatim}[commandchars=\\\{\}]
{\color{incolor}In [{\color{incolor} }]:} \PY{k+kn}{from} \PY{n+nn}{keras}\PY{n+nn}{.}\PY{n+nn}{callbacks} \PY{k}{import} \PY{n}{ModelCheckpoint}
        \PY{k+kn}{from} \PY{n+nn}{keras}\PY{n+nn}{.}\PY{n+nn}{models} \PY{k}{import} \PY{n}{Model}\PY{p}{,} \PY{n}{load\PYZus{}model}\PY{p}{,} \PY{n}{Sequential}
        \PY{k+kn}{from} \PY{n+nn}{keras}\PY{n+nn}{.}\PY{n+nn}{layers} \PY{k}{import} \PY{n}{Dense}\PY{p}{,} \PY{n}{Activation}\PY{p}{,} \PY{n}{Dropout}\PY{p}{,} \PY{n}{Input}\PY{p}{,} \PY{n}{Masking}\PY{p}{,} \PY{n}{TimeDistributed}\PY{p}{,} \PY{n}{LSTM}\PY{p}{,} \PY{n}{Conv1D}
        \PY{k+kn}{from} \PY{n+nn}{keras}\PY{n+nn}{.}\PY{n+nn}{layers} \PY{k}{import} \PY{n}{GRU}\PY{p}{,} \PY{n}{Bidirectional}\PY{p}{,} \PY{n}{BatchNormalization}\PY{p}{,} \PY{n}{Reshape}
        \PY{k+kn}{from} \PY{n+nn}{keras}\PY{n+nn}{.}\PY{n+nn}{optimizers} \PY{k}{import} \PY{n}{Adam}
\end{Verbatim}

    \subsection{2.1 - Build the model}\label{build-the-model}

Our goal is to build a network that will ingest a spectrogram and output
a signal when it detects the trigger word. This network will use 4
layers: * A convolutional layer * Two GRU layers * A dense layer.

Here is the architecture we will use.

\textbf{Figure 3}

\subparagraph{1D convolutional layer}\label{d-convolutional-layer}

One key layer of this model is the 1D convolutional step (near the
bottom of Figure 3). * It inputs the 5511 step spectrogram. Each step is
a vector of 101 units. * It outputs a 1375 step output * This output is
further processed by multiple layers to get the final $T_y = 1375$ step
output. * This 1D convolutional layer plays a role similar to the 2D
convolutions you saw in Course 4, of extracting low-level features and
then possibly generating an output of a smaller dimension. *
Computationally, the 1-D conv layer also helps speed up the model
because now the GRU can process only 1375 timesteps rather than 5511
timesteps.

\subparagraph{GRU, dense and sigmoid}\label{gru-dense-and-sigmoid}

\begin{itemize}
\itemsep1pt\parskip0pt\parsep0pt
\item
  The two GRU layers read the sequence of inputs from left to right.
\item
  A dense plus sigmoid layer makes a prediction for
  $y^{\langle t \rangle}$.
\item
  Because $y$ is a binary value (0 or 1), we use a sigmoid output at the
  last layer to estimate the chance of the output being 1, corresponding
  to the user having just said ``activate.''
\end{itemize}

\paragraph{Unidirectional RNN}\label{unidirectional-rnn}

\begin{itemize}
\itemsep1pt\parskip0pt\parsep0pt
\item
  Note that we use a \textbf{unidirectional RNN} rather than a
  bidirectional RNN.
\item
  This is really important for trigger word detection, since we want to
  be able to detect the trigger word almost immediately after it is
  said.
\item
  If we used a bidirectional RNN, we would have to wait for the whole
  10sec of audio to be recorded before we could tell if ``activate'' was
  said in the first second of the audio clip.
\end{itemize}

    \paragraph{Implement the model}\label{implement-the-model}

Implementing the model can be done in four steps:

\textbf{Step 1}: CONV layer. Use \texttt{Conv1D()} to implement this,
with 196 filters, a filter size of 15 (\texttt{kernel\_size=15}), and
stride of 4.
\href{https://keras.io/layers/convolutional/\#conv1d}{conv1d}

\begin{Shaded}
\begin{Highlighting}[]
\NormalTok{output_x = Conv1D(filters=...,kernel_size=...,strides=...)(input_x)}
\end{Highlighting}
\end{Shaded}

\begin{itemize}
\itemsep1pt\parskip0pt\parsep0pt
\item
  Follow this with a ReLu activation. Note that we can pass in the name
  of the desired activation as a string, all in lowercase letters.
\end{itemize}

\begin{Shaded}
\begin{Highlighting}[]
\NormalTok{output_x = Activation(}\StringTok{"..."}\NormalTok{)(input_x)}
\end{Highlighting}
\end{Shaded}

\begin{itemize}
\itemsep1pt\parskip0pt\parsep0pt
\item
  Follow this with dropout, using a keep rate of 0.8
\end{itemize}

\begin{Shaded}
\begin{Highlighting}[]
\NormalTok{output_x = Dropout(rate=...)(input_x)}
\end{Highlighting}
\end{Shaded}

\textbf{Step 2}: First GRU layer. To generate the GRU layer, use 128
units.

\begin{Shaded}
\begin{Highlighting}[]
\NormalTok{output_x = GRU(units=..., return_sequences = ...)(input_x)}
\end{Highlighting}
\end{Shaded}

\begin{itemize}
\item
  Return sequences instead of just the last time step's prediction to
  ensures that all the GRU's hidden states are fed to the next layer.
\item
  Follow this with dropout, using a keep rate of 0.8.
\item
  Follow this with batch normalization. No parameters need to be set.

\begin{Shaded}
\begin{Highlighting}[]
\NormalTok{output_x = BatchNormalization()(input_x)}
\end{Highlighting}
\end{Shaded}
\end{itemize}

\textbf{Step 3}: Second GRU layer. This has the same specifications as
the first GRU layer. * Follow this with a dropout, batch normalization,
and then another dropout.

\textbf{Step 4}: Create a time-distributed dense layer as follows:

\begin{Shaded}
\begin{Highlighting}[]
\NormalTok{X = TimeDistributed(Dense(}\DecValTok{1}\NormalTok{, activation = }\StringTok{"sigmoid"}\NormalTok{))(X)}
\end{Highlighting}
\end{Shaded}

This creates a dense layer followed by a sigmoid, so that the parameters
used for the dense layer are the same for every time
step.\\Documentation: * \href{https://keras.io/layers/wrappers/}{Keras
documentation on wrappers}.\\* To learn more, you can read this blog
post
\href{https://machinelearningmastery.com/timedistributed-layer-for-long-short-term-memory-networks-in-python/}{How
to Use the TimeDistributed Layer in Keras}.

\textbf{Exercise}: Implement \texttt{model()}, the architecture is
presented in Figure 3.

    \begin{Verbatim}[commandchars=\\\{\}]
{\color{incolor}In [{\color{incolor} }]:} \PY{c+c1}{\PYZsh{} GRADED FUNCTION: model}
        
        \PY{k}{def} \PY{n+nf}{model}\PY{p}{(}\PY{n}{input\PYZus{}shape}\PY{p}{)}\PY{p}{:}
            \PY{l+s+sd}{\PYZdq{}\PYZdq{}\PYZdq{}}
        \PY{l+s+sd}{    Function creating the model\PYZsq{}s graph in Keras.}
        \PY{l+s+sd}{    }
        \PY{l+s+sd}{    Argument:}
        \PY{l+s+sd}{    input\PYZus{}shape \PYZhy{}\PYZhy{} shape of the model\PYZsq{}s input data (using Keras conventions)}
        
        \PY{l+s+sd}{    Returns:}
        \PY{l+s+sd}{    model \PYZhy{}\PYZhy{} Keras model instance}
        \PY{l+s+sd}{    \PYZdq{}\PYZdq{}\PYZdq{}}
            
            \PY{n}{X\PYZus{}input} \PY{o}{=} \PY{n}{Input}\PY{p}{(}\PY{n}{shape} \PY{o}{=} \PY{n}{input\PYZus{}shape}\PY{p}{)}
            
            \PY{c+c1}{\PYZsh{}\PYZsh{}\PYZsh{} START CODE HERE \PYZsh{}\PYZsh{}\PYZsh{}}
            
            \PY{c+c1}{\PYZsh{} Step 1: CONV layer (≈4 lines)}
            \PY{n}{X} \PY{o}{=} \PY{k+kc}{None}                                 \PY{c+c1}{\PYZsh{} CONV1D}
            \PY{n}{X} \PY{o}{=} \PY{k+kc}{None}                                 \PY{c+c1}{\PYZsh{} Batch normalization}
            \PY{n}{X} \PY{o}{=} \PY{k+kc}{None}                                 \PY{c+c1}{\PYZsh{} ReLu activation}
            \PY{n}{X} \PY{o}{=} \PY{k+kc}{None}                                 \PY{c+c1}{\PYZsh{} dropout (use 0.8)}
        
            \PY{c+c1}{\PYZsh{} Step 2: First GRU Layer (≈4 lines)}
            \PY{n}{X} \PY{o}{=} \PY{k+kc}{None}                                 \PY{c+c1}{\PYZsh{} GRU (use 128 units and return the sequences)}
            \PY{n}{X} \PY{o}{=} \PY{k+kc}{None}                                 \PY{c+c1}{\PYZsh{} dropout (use 0.8)}
            \PY{n}{X} \PY{o}{=} \PY{k+kc}{None}                                 \PY{c+c1}{\PYZsh{} Batch normalization}
            
            \PY{c+c1}{\PYZsh{} Step 3: Second GRU Layer (≈4 lines)}
            \PY{n}{X} \PY{o}{=} \PY{k+kc}{None}                                 \PY{c+c1}{\PYZsh{} GRU (use 128 units and return the sequences)}
            \PY{n}{X} \PY{o}{=} \PY{k+kc}{None}                                 \PY{c+c1}{\PYZsh{} dropout (use 0.8)}
            \PY{n}{X} \PY{o}{=} \PY{k+kc}{None}                                 \PY{c+c1}{\PYZsh{} Batch normalization}
            \PY{n}{X} \PY{o}{=} \PY{k+kc}{None}                                 \PY{c+c1}{\PYZsh{} dropout (use 0.8)}
            
            \PY{c+c1}{\PYZsh{} Step 4: Time\PYZhy{}distributed dense layer (see given code in instructions) (≈1 line)}
            \PY{n}{X} \PY{o}{=} \PY{k+kc}{None} \PY{c+c1}{\PYZsh{} time distributed  (sigmoid)}
        
            \PY{c+c1}{\PYZsh{}\PYZsh{}\PYZsh{} END CODE HERE \PYZsh{}\PYZsh{}\PYZsh{}}
        
            \PY{n}{model} \PY{o}{=} \PY{n}{Model}\PY{p}{(}\PY{n}{inputs} \PY{o}{=} \PY{n}{X\PYZus{}input}\PY{p}{,} \PY{n}{outputs} \PY{o}{=} \PY{n}{X}\PY{p}{)}
            
            \PY{k}{return} \PY{n}{model}  
\end{Verbatim}

    \begin{Verbatim}[commandchars=\\\{\}]
{\color{incolor}In [{\color{incolor} }]:} \PY{n}{model} \PY{o}{=} \PY{n}{model}\PY{p}{(}\PY{n}{input\PYZus{}shape} \PY{o}{=} \PY{p}{(}\PY{n}{Tx}\PY{p}{,} \PY{n}{n\PYZus{}freq}\PY{p}{)}\PY{p}{)}
\end{Verbatim}

    Let's print the model summary to keep track of the shapes.

    \begin{Verbatim}[commandchars=\\\{\}]
{\color{incolor}In [{\color{incolor} }]:} \PY{n}{model}\PY{o}{.}\PY{n}{summary}\PY{p}{(}\PY{p}{)}
\end{Verbatim}

    \textbf{Expected Output}:

\textbf{Total params}

522,561

\textbf{Trainable params}

521,657

\textbf{Non-trainable params}

904

    The output of the network is of shape (None, 1375, 1) while the input is
(None, 5511, 101). The Conv1D has reduced the number of steps from 5511
to 1375.

    \subsection{2.2 - Fit the model}\label{fit-the-model}

    \begin{itemize}
\itemsep1pt\parskip0pt\parsep0pt
\item
  Trigger word detection takes a long time to train.
\item
  To save time, we've already trained a model for about 3 hours on a GPU
  using the architecture you built above, and a large training set of
  about 4000 examples.
\item
  Let's load the model.
\end{itemize}

    \begin{Verbatim}[commandchars=\\\{\}]
{\color{incolor}In [{\color{incolor} }]:} \PY{n}{model} \PY{o}{=} \PY{n}{load\PYZus{}model}\PY{p}{(}\PY{l+s+s1}{\PYZsq{}}\PY{l+s+s1}{./models/tr\PYZus{}model.h5}\PY{l+s+s1}{\PYZsq{}}\PY{p}{)}
\end{Verbatim}

    You can train the model further, using the Adam optimizer and binary
cross entropy loss, as follows. This will run quickly because we are
training just for one epoch and with a small training set of 26
examples.

    \begin{Verbatim}[commandchars=\\\{\}]
{\color{incolor}In [{\color{incolor} }]:} \PY{n}{opt} \PY{o}{=} \PY{n}{Adam}\PY{p}{(}\PY{n}{lr}\PY{o}{=}\PY{l+m+mf}{0.0001}\PY{p}{,} \PY{n}{beta\PYZus{}1}\PY{o}{=}\PY{l+m+mf}{0.9}\PY{p}{,} \PY{n}{beta\PYZus{}2}\PY{o}{=}\PY{l+m+mf}{0.999}\PY{p}{,} \PY{n}{decay}\PY{o}{=}\PY{l+m+mf}{0.01}\PY{p}{)}
        \PY{n}{model}\PY{o}{.}\PY{n}{compile}\PY{p}{(}\PY{n}{loss}\PY{o}{=}\PY{l+s+s1}{\PYZsq{}}\PY{l+s+s1}{binary\PYZus{}crossentropy}\PY{l+s+s1}{\PYZsq{}}\PY{p}{,} \PY{n}{optimizer}\PY{o}{=}\PY{n}{opt}\PY{p}{,} \PY{n}{metrics}\PY{o}{=}\PY{p}{[}\PY{l+s+s2}{\PYZdq{}}\PY{l+s+s2}{accuracy}\PY{l+s+s2}{\PYZdq{}}\PY{p}{]}\PY{p}{)}
\end{Verbatim}

    \begin{Verbatim}[commandchars=\\\{\}]
{\color{incolor}In [{\color{incolor} }]:} \PY{n}{model}\PY{o}{.}\PY{n}{fit}\PY{p}{(}\PY{n}{X}\PY{p}{,} \PY{n}{Y}\PY{p}{,} \PY{n}{batch\PYZus{}size} \PY{o}{=} \PY{l+m+mi}{5}\PY{p}{,} \PY{n}{epochs}\PY{o}{=}\PY{l+m+mi}{1}\PY{p}{)}
\end{Verbatim}

    \subsection{2.3 - Test the model}\label{test-the-model}

Finally, let's see how your model performs on the dev set.

    \begin{Verbatim}[commandchars=\\\{\}]
{\color{incolor}In [{\color{incolor} }]:} \PY{n}{loss}\PY{p}{,} \PY{n}{acc} \PY{o}{=} \PY{n}{model}\PY{o}{.}\PY{n}{evaluate}\PY{p}{(}\PY{n}{X\PYZus{}dev}\PY{p}{,} \PY{n}{Y\PYZus{}dev}\PY{p}{)}
        \PY{n+nb}{print}\PY{p}{(}\PY{l+s+s2}{\PYZdq{}}\PY{l+s+s2}{Dev set accuracy = }\PY{l+s+s2}{\PYZdq{}}\PY{p}{,} \PY{n}{acc}\PY{p}{)}
\end{Verbatim}

    This looks pretty good! * However, accuracy isn't a great metric for
this task * Since the labels are heavily skewed to 0's, a neural network
that just outputs 0's would get slightly over 90\% accuracy. * We could
define more useful metrics such as F1 score or Precision/Recall. * Let's
not bother with that here, and instead just empirically see how the
model does with some predictions.

    \section{3 - Making Predictions}\label{making-predictions}

Now that you have built a working model for trigger word detection,
let's use it to make predictions. This code snippet runs audio (saved in
a wav file) through the network.

    \begin{Verbatim}[commandchars=\\\{\}]
{\color{incolor}In [{\color{incolor} }]:} \PY{k}{def} \PY{n+nf}{detect\PYZus{}triggerword}\PY{p}{(}\PY{n}{filename}\PY{p}{)}\PY{p}{:}
            \PY{n}{plt}\PY{o}{.}\PY{n}{subplot}\PY{p}{(}\PY{l+m+mi}{2}\PY{p}{,} \PY{l+m+mi}{1}\PY{p}{,} \PY{l+m+mi}{1}\PY{p}{)}
        
            \PY{n}{x} \PY{o}{=} \PY{n}{graph\PYZus{}spectrogram}\PY{p}{(}\PY{n}{filename}\PY{p}{)}
            \PY{c+c1}{\PYZsh{} the spectrogram outputs (freqs, Tx) and we want (Tx, freqs) to input into the model}
            \PY{n}{x}  \PY{o}{=} \PY{n}{x}\PY{o}{.}\PY{n}{swapaxes}\PY{p}{(}\PY{l+m+mi}{0}\PY{p}{,}\PY{l+m+mi}{1}\PY{p}{)}
            \PY{n}{x} \PY{o}{=} \PY{n}{np}\PY{o}{.}\PY{n}{expand\PYZus{}dims}\PY{p}{(}\PY{n}{x}\PY{p}{,} \PY{n}{axis}\PY{o}{=}\PY{l+m+mi}{0}\PY{p}{)}
            \PY{n}{predictions} \PY{o}{=} \PY{n}{model}\PY{o}{.}\PY{n}{predict}\PY{p}{(}\PY{n}{x}\PY{p}{)}
            
            \PY{n}{plt}\PY{o}{.}\PY{n}{subplot}\PY{p}{(}\PY{l+m+mi}{2}\PY{p}{,} \PY{l+m+mi}{1}\PY{p}{,} \PY{l+m+mi}{2}\PY{p}{)}
            \PY{n}{plt}\PY{o}{.}\PY{n}{plot}\PY{p}{(}\PY{n}{predictions}\PY{p}{[}\PY{l+m+mi}{0}\PY{p}{,}\PY{p}{:}\PY{p}{,}\PY{l+m+mi}{0}\PY{p}{]}\PY{p}{)}
            \PY{n}{plt}\PY{o}{.}\PY{n}{ylabel}\PY{p}{(}\PY{l+s+s1}{\PYZsq{}}\PY{l+s+s1}{probability}\PY{l+s+s1}{\PYZsq{}}\PY{p}{)}
            \PY{n}{plt}\PY{o}{.}\PY{n}{show}\PY{p}{(}\PY{p}{)}
            \PY{k}{return} \PY{n}{predictions}
\end{Verbatim}

    \paragraph{Insert a chime to acknowledge the ``activate''
trigger}\label{insert-a-chime-to-acknowledge-the-activate-trigger}

\begin{itemize}
\itemsep1pt\parskip0pt\parsep0pt
\item
  Once you've estimated the probability of having detected the word
  ``activate'' at each output step, you can trigger a ``chiming'' sound
  to play when the probability is above a certain threshold.
\item
  $y^{\langle t \rangle}$ might be near 1 for many values in a row after
  ``activate'' is said, yet we want to chime only once.

  \begin{itemize}
  \itemsep1pt\parskip0pt\parsep0pt
  \item
    So we will insert a chime sound at most once every 75 output steps.
  \item
    This will help prevent us from inserting two chimes for a single
    instance of ``activate''.
  \item
    This plays a role similar to non-max suppression from computer
    vision.
  \end{itemize}
\end{itemize}

    \begin{Verbatim}[commandchars=\\\{\}]
{\color{incolor}In [{\color{incolor} }]:} \PY{n}{chime\PYZus{}file} \PY{o}{=} \PY{l+s+s2}{\PYZdq{}}\PY{l+s+s2}{audio\PYZus{}examples/chime.wav}\PY{l+s+s2}{\PYZdq{}}
        \PY{k}{def} \PY{n+nf}{chime\PYZus{}on\PYZus{}activate}\PY{p}{(}\PY{n}{filename}\PY{p}{,} \PY{n}{predictions}\PY{p}{,} \PY{n}{threshold}\PY{p}{)}\PY{p}{:}
            \PY{n}{audio\PYZus{}clip} \PY{o}{=} \PY{n}{AudioSegment}\PY{o}{.}\PY{n}{from\PYZus{}wav}\PY{p}{(}\PY{n}{filename}\PY{p}{)}
            \PY{n}{chime} \PY{o}{=} \PY{n}{AudioSegment}\PY{o}{.}\PY{n}{from\PYZus{}wav}\PY{p}{(}\PY{n}{chime\PYZus{}file}\PY{p}{)}
            \PY{n}{Ty} \PY{o}{=} \PY{n}{predictions}\PY{o}{.}\PY{n}{shape}\PY{p}{[}\PY{l+m+mi}{1}\PY{p}{]}
            \PY{c+c1}{\PYZsh{} Step 1: Initialize the number of consecutive output steps to 0}
            \PY{n}{consecutive\PYZus{}timesteps} \PY{o}{=} \PY{l+m+mi}{0}
            \PY{c+c1}{\PYZsh{} Step 2: Loop over the output steps in the y}
            \PY{k}{for} \PY{n}{i} \PY{o+ow}{in} \PY{n+nb}{range}\PY{p}{(}\PY{n}{Ty}\PY{p}{)}\PY{p}{:}
                \PY{c+c1}{\PYZsh{} Step 3: Increment consecutive output steps}
                \PY{n}{consecutive\PYZus{}timesteps} \PY{o}{+}\PY{o}{=} \PY{l+m+mi}{1}
                \PY{c+c1}{\PYZsh{} Step 4: If prediction is higher than the threshold and more than 75 consecutive output steps have passed}
                \PY{k}{if} \PY{n}{predictions}\PY{p}{[}\PY{l+m+mi}{0}\PY{p}{,}\PY{n}{i}\PY{p}{,}\PY{l+m+mi}{0}\PY{p}{]} \PY{o}{\PYZgt{}} \PY{n}{threshold} \PY{o+ow}{and} \PY{n}{consecutive\PYZus{}timesteps} \PY{o}{\PYZgt{}} \PY{l+m+mi}{75}\PY{p}{:}
                    \PY{c+c1}{\PYZsh{} Step 5: Superpose audio and background using pydub}
                    \PY{n}{audio\PYZus{}clip} \PY{o}{=} \PY{n}{audio\PYZus{}clip}\PY{o}{.}\PY{n}{overlay}\PY{p}{(}\PY{n}{chime}\PY{p}{,} \PY{n}{position} \PY{o}{=} \PY{p}{(}\PY{p}{(}\PY{n}{i} \PY{o}{/} \PY{n}{Ty}\PY{p}{)} \PY{o}{*} \PY{n}{audio\PYZus{}clip}\PY{o}{.}\PY{n}{duration\PYZus{}seconds}\PY{p}{)}\PY{o}{*}\PY{l+m+mi}{1000}\PY{p}{)}
                    \PY{c+c1}{\PYZsh{} Step 6: Reset consecutive output steps to 0}
                    \PY{n}{consecutive\PYZus{}timesteps} \PY{o}{=} \PY{l+m+mi}{0}
                
            \PY{n}{audio\PYZus{}clip}\PY{o}{.}\PY{n}{export}\PY{p}{(}\PY{l+s+s2}{\PYZdq{}}\PY{l+s+s2}{chime\PYZus{}output.wav}\PY{l+s+s2}{\PYZdq{}}\PY{p}{,} \PY{n+nb}{format}\PY{o}{=}\PY{l+s+s1}{\PYZsq{}}\PY{l+s+s1}{wav}\PY{l+s+s1}{\PYZsq{}}\PY{p}{)}
\end{Verbatim}

    \subsection{3.3 - Test on dev examples}\label{test-on-dev-examples}

    Let's explore how our model performs on two unseen audio clips from the
development set. Lets first listen to the two dev set clips.

    \begin{Verbatim}[commandchars=\\\{\}]
{\color{incolor}In [{\color{incolor} }]:} \PY{n}{IPython}\PY{o}{.}\PY{n}{display}\PY{o}{.}\PY{n}{Audio}\PY{p}{(}\PY{l+s+s2}{\PYZdq{}}\PY{l+s+s2}{./raw\PYZus{}data/dev/1.wav}\PY{l+s+s2}{\PYZdq{}}\PY{p}{)}
\end{Verbatim}

    \begin{Verbatim}[commandchars=\\\{\}]
{\color{incolor}In [{\color{incolor} }]:} \PY{n}{IPython}\PY{o}{.}\PY{n}{display}\PY{o}{.}\PY{n}{Audio}\PY{p}{(}\PY{l+s+s2}{\PYZdq{}}\PY{l+s+s2}{./raw\PYZus{}data/dev/2.wav}\PY{l+s+s2}{\PYZdq{}}\PY{p}{)}
\end{Verbatim}

    Now lets run the model on these audio clips and see if it adds a chime
after ``activate''!

    \begin{Verbatim}[commandchars=\\\{\}]
{\color{incolor}In [{\color{incolor} }]:} \PY{n}{filename} \PY{o}{=} \PY{l+s+s2}{\PYZdq{}}\PY{l+s+s2}{./raw\PYZus{}data/dev/1.wav}\PY{l+s+s2}{\PYZdq{}}
        \PY{n}{prediction} \PY{o}{=} \PY{n}{detect\PYZus{}triggerword}\PY{p}{(}\PY{n}{filename}\PY{p}{)}
        \PY{n}{chime\PYZus{}on\PYZus{}activate}\PY{p}{(}\PY{n}{filename}\PY{p}{,} \PY{n}{prediction}\PY{p}{,} \PY{l+m+mf}{0.5}\PY{p}{)}
        \PY{n}{IPython}\PY{o}{.}\PY{n}{display}\PY{o}{.}\PY{n}{Audio}\PY{p}{(}\PY{l+s+s2}{\PYZdq{}}\PY{l+s+s2}{./chime\PYZus{}output.wav}\PY{l+s+s2}{\PYZdq{}}\PY{p}{)}
\end{Verbatim}

    \begin{Verbatim}[commandchars=\\\{\}]
{\color{incolor}In [{\color{incolor} }]:} \PY{n}{filename}  \PY{o}{=} \PY{l+s+s2}{\PYZdq{}}\PY{l+s+s2}{./raw\PYZus{}data/dev/2.wav}\PY{l+s+s2}{\PYZdq{}}
        \PY{n}{prediction} \PY{o}{=} \PY{n}{detect\PYZus{}triggerword}\PY{p}{(}\PY{n}{filename}\PY{p}{)}
        \PY{n}{chime\PYZus{}on\PYZus{}activate}\PY{p}{(}\PY{n}{filename}\PY{p}{,} \PY{n}{prediction}\PY{p}{,} \PY{l+m+mf}{0.5}\PY{p}{)}
        \PY{n}{IPython}\PY{o}{.}\PY{n}{display}\PY{o}{.}\PY{n}{Audio}\PY{p}{(}\PY{l+s+s2}{\PYZdq{}}\PY{l+s+s2}{./chime\PYZus{}output.wav}\PY{l+s+s2}{\PYZdq{}}\PY{p}{)}
\end{Verbatim}

    \section{Congratulations}\label{congratulations}

You've come to the end of this assignment!

\subsection{Here's what you should
remember:}\label{heres-what-you-should-remember}

\begin{itemize}
\itemsep1pt\parskip0pt\parsep0pt
\item
  Data synthesis is an effective way to create a large training set for
  speech problems, specifically trigger word detection.
\item
  Using a spectrogram and optionally a 1D conv layer is a common
  pre-processing step prior to passing audio data to an RNN, GRU or
  LSTM.
\item
  An end-to-end deep learning approach can be used to build a very
  effective trigger word detection system.
\end{itemize}

\emph{Congratulations} on finishing the final assignment!

Thank you for sticking with us through the end and for all the hard work
you've put into learning deep learning. We hope you have enjoyed the
course!

    \section{4 - Try your own example!
(OPTIONAL/UNGRADED)}\label{try-your-own-example-optionalungraded}

In this optional and ungraded portion of this notebook, you can try your
model on your own audio clips!

\begin{itemize}
\itemsep1pt\parskip0pt\parsep0pt
\item
  Record a 10 second audio clip of you saying the word ``activate'' and
  other random words, and upload it to the Coursera hub as
  \texttt{myaudio.wav}.
\item
  Be sure to upload the audio as a wav file.
\item
  If your audio is recorded in a different format (such as mp3) there is
  free software that you can find online for converting it to wav.
\item
  If your audio recording is not 10 seconds, the code below will either
  trim or pad it as needed to make it 10 seconds.
\end{itemize}

    \begin{Verbatim}[commandchars=\\\{\}]
{\color{incolor}In [{\color{incolor} }]:} \PY{c+c1}{\PYZsh{} Preprocess the audio to the correct format}
        \PY{k}{def} \PY{n+nf}{preprocess\PYZus{}audio}\PY{p}{(}\PY{n}{filename}\PY{p}{)}\PY{p}{:}
            \PY{c+c1}{\PYZsh{} Trim or pad audio segment to 10000ms}
            \PY{n}{padding} \PY{o}{=} \PY{n}{AudioSegment}\PY{o}{.}\PY{n}{silent}\PY{p}{(}\PY{n}{duration}\PY{o}{=}\PY{l+m+mi}{10000}\PY{p}{)}
            \PY{n}{segment} \PY{o}{=} \PY{n}{AudioSegment}\PY{o}{.}\PY{n}{from\PYZus{}wav}\PY{p}{(}\PY{n}{filename}\PY{p}{)}\PY{p}{[}\PY{p}{:}\PY{l+m+mi}{10000}\PY{p}{]}
            \PY{n}{segment} \PY{o}{=} \PY{n}{padding}\PY{o}{.}\PY{n}{overlay}\PY{p}{(}\PY{n}{segment}\PY{p}{)}
            \PY{c+c1}{\PYZsh{} Set frame rate to 44100}
            \PY{n}{segment} \PY{o}{=} \PY{n}{segment}\PY{o}{.}\PY{n}{set\PYZus{}frame\PYZus{}rate}\PY{p}{(}\PY{l+m+mi}{44100}\PY{p}{)}
            \PY{c+c1}{\PYZsh{} Export as wav}
            \PY{n}{segment}\PY{o}{.}\PY{n}{export}\PY{p}{(}\PY{n}{filename}\PY{p}{,} \PY{n+nb}{format}\PY{o}{=}\PY{l+s+s1}{\PYZsq{}}\PY{l+s+s1}{wav}\PY{l+s+s1}{\PYZsq{}}\PY{p}{)}
\end{Verbatim}

    Once you've uploaded your audio file to Coursera, put the path to your
file in the variable below.

    \begin{Verbatim}[commandchars=\\\{\}]
{\color{incolor}In [{\color{incolor} }]:} \PY{n}{your\PYZus{}filename} \PY{o}{=} \PY{l+s+s2}{\PYZdq{}}\PY{l+s+s2}{audio\PYZus{}examples/my\PYZus{}audio.wav}\PY{l+s+s2}{\PYZdq{}}
\end{Verbatim}

    \begin{Verbatim}[commandchars=\\\{\}]
{\color{incolor}In [{\color{incolor} }]:} \PY{n}{preprocess\PYZus{}audio}\PY{p}{(}\PY{n}{your\PYZus{}filename}\PY{p}{)}
        \PY{n}{IPython}\PY{o}{.}\PY{n}{display}\PY{o}{.}\PY{n}{Audio}\PY{p}{(}\PY{n}{your\PYZus{}filename}\PY{p}{)} \PY{c+c1}{\PYZsh{} listen to the audio you uploaded }
\end{Verbatim}

    Finally, use the model to predict when you say activate in the 10 second
audio clip, and trigger a chime. If beeps are not being added
appropriately, try to adjust the chime\_threshold.

    \begin{Verbatim}[commandchars=\\\{\}]
{\color{incolor}In [{\color{incolor} }]:} \PY{n}{chime\PYZus{}threshold} \PY{o}{=} \PY{l+m+mf}{0.5}
        \PY{n}{prediction} \PY{o}{=} \PY{n}{detect\PYZus{}triggerword}\PY{p}{(}\PY{n}{your\PYZus{}filename}\PY{p}{)}
        \PY{n}{chime\PYZus{}on\PYZus{}activate}\PY{p}{(}\PY{n}{your\PYZus{}filename}\PY{p}{,} \PY{n}{prediction}\PY{p}{,} \PY{n}{chime\PYZus{}threshold}\PY{p}{)}
        \PY{n}{IPython}\PY{o}{.}\PY{n}{display}\PY{o}{.}\PY{n}{Audio}\PY{p}{(}\PY{l+s+s2}{\PYZdq{}}\PY{l+s+s2}{./chime\PYZus{}output.wav}\PY{l+s+s2}{\PYZdq{}}\PY{p}{)}
\end{Verbatim}


    % Add a bibliography block to the postdoc
    
    
    
    \end{document}
